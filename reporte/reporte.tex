%%%%%%%%%%%%%%%%%%%%%%%%%%%%%%%%%%%%%%%%%
% Simple Sectioned Essay Template
% LaTeX Template
%
% This template has been downloaded from:
% http://www.latextemplates.com
%
% Note:
% The \lipsum[#] commands throughout this template generate dummy text
% to fill the template out. These commands should all be removed when
% writing essay content.
%
%%%%%%%%%%%%%%%%%%%%%%%%%%%%%%%%%%%%%%%%%

%----------------------------------------------------------------------------------------
%	PACKAGES AND OTHER DOCUMENT CONFIGURATIONS
%----------------------------------------------------------------------------------------

\documentclass[11pt]{article}

    \usepackage[T1]{fontenc}


    % Basic figure setup, for now with no caption control since it's done
    % automatically by Pandoc (which extracts ![](path) syntax from Markdown).
    \usepackage{graphicx}
    % We will generate all images so they have a width \maxwidth. This means
    % that they will get their normal width if they fit onto the page, but
    % are scaled down if they would overflow the margins.
    \makeatletter
    \def\maxwidth{\ifdim\Gin@nat@width>\linewidth\linewidth
    \else\Gin@nat@width\fi}
    \makeatother
    \let\Oldincludegraphics\includegraphics
    % Set max figure width to be 80% of text width, for now hardcoded.
    \renewcommand{\includegraphics}[1]{\Oldincludegraphics[width=.8\maxwidth]{#1}}
    % Ensure that by default, figures have no caption (until we provide a
    % proper Figure object with a Caption API and a way to capture that
    % in the conversion process - todo).
    \usepackage{caption}
    \DeclareCaptionLabelFormat{nolabel}{}
    \captionsetup{labelformat=nolabel}

    \usepackage{adjustbox} % Used to constrain images to a maximum size
    \usepackage{xcolor} % Allow colors to be defined
    \usepackage{enumerate} % Needed for markdown enumerations to work
    \usepackage{geometry} % Used to adjust the document margins
    \usepackage{amsmath} % Equations
    \usepackage{amssymb} % Equations
    \usepackage{textcomp} % defines textquotesingle
    % Hack from http://tex.stackexchange.com/a/47451/13684:
    \AtBeginDocument{%
        \def\PYZsq{\textquotesingle}% Upright quotes in Pygmentized code
    }
    \usepackage{upquote} % Upright quotes for verbatim code
    \usepackage{eurosym} % defines \euro
    \usepackage[mathletters]{ucs} % Extended unicode (utf-8) support
    \usepackage[utf8x]{inputenc} % Allow utf-8 characters in the tex document
    \usepackage{fancyvrb} % verbatim replacement that allows latex
    \usepackage{grffile} % extends the file name processing of package graphics
                         % to support a larger range
    % The hyperref package gives us a pdf with properly built
    % internal navigation ('pdf bookmarks' for the table of contents,
    % internal cross-reference links, web links for URLs, etc.)
    \usepackage{hyperref}
    \usepackage{longtable} % longtable support required by pandoc >1.10
    \usepackage{booktabs}  % table support for pandoc > 1.12.2
    \usepackage[inline]{enumitem} % IRkernel/repr support (it uses the enumerate* environment)
    \usepackage[normalem]{ulem} % ulem is needed to support strikethroughs (\sout)
                                % normalem makes italics be italics, not underlines




    % Colors for the hyperref package
    \definecolor{urlcolor}{rgb}{0,.145,.698}
    \definecolor{linkcolor}{rgb}{.71,0.21,0.01}
    \definecolor{citecolor}{rgb}{.12,.54,.11}

    % ANSI colors
    \definecolor{ansi-black}{HTML}{3E424D}
    \definecolor{ansi-black-intense}{HTML}{282C36}
    \definecolor{ansi-red}{HTML}{E75C58}
    \definecolor{ansi-red-intense}{HTML}{B22B31}
    \definecolor{ansi-green}{HTML}{00A250}
    \definecolor{ansi-green-intense}{HTML}{007427}
    \definecolor{ansi-yellow}{HTML}{DDB62B}
    \definecolor{ansi-yellow-intense}{HTML}{B27D12}
    \definecolor{ansi-blue}{HTML}{208FFB}
    \definecolor{ansi-blue-intense}{HTML}{0065CA}
    \definecolor{ansi-magenta}{HTML}{D160C4}
    \definecolor{ansi-magenta-intense}{HTML}{A03196}
    \definecolor{ansi-cyan}{HTML}{60C6C8}
    \definecolor{ansi-cyan-intense}{HTML}{258F8F}
    \definecolor{ansi-white}{HTML}{C5C1B4}
    \definecolor{ansi-white-intense}{HTML}{A1A6B2}

    % commands and environments needed by pandoc snippets
    % extracted from the output of `pandoc -s`
    \providecommand{\tightlist}{%
      \setlength{\itemsep}{0pt}\setlength{\parskip}{0pt}}
    \DefineVerbatimEnvironment{Highlighting}{Verbatim}{commandchars=\\\{\}}
    % Add ',fontsize=\small' for more characters per line
    \newenvironment{Shaded}{}{}
    \newcommand{\KeywordTok}[1]{\textcolor[rgb]{0.00,0.44,0.13}{\textbf{{#1}}}}
    \newcommand{\DataTypeTok}[1]{\textcolor[rgb]{0.56,0.13,0.00}{{#1}}}
    \newcommand{\DecValTok}[1]{\textcolor[rgb]{0.25,0.63,0.44}{{#1}}}
    \newcommand{\BaseNTok}[1]{\textcolor[rgb]{0.25,0.63,0.44}{{#1}}}
    \newcommand{\FloatTok}[1]{\textcolor[rgb]{0.25,0.63,0.44}{{#1}}}
    \newcommand{\CharTok}[1]{\textcolor[rgb]{0.25,0.44,0.63}{{#1}}}
    \newcommand{\StringTok}[1]{\textcolor[rgb]{0.25,0.44,0.63}{{#1}}}
    \newcommand{\CommentTok}[1]{\textcolor[rgb]{0.38,0.63,0.69}{\textit{{#1}}}}
    \newcommand{\OtherTok}[1]{\textcolor[rgb]{0.00,0.44,0.13}{{#1}}}
    \newcommand{\AlertTok}[1]{\textcolor[rgb]{1.00,0.00,0.00}{\textbf{{#1}}}}
    \newcommand{\FunctionTok}[1]{\textcolor[rgb]{0.02,0.16,0.49}{{#1}}}
    \newcommand{\RegionMarkerTok}[1]{{#1}}
    \newcommand{\ErrorTok}[1]{\textcolor[rgb]{1.00,0.00,0.00}{\textbf{{#1}}}}
    \newcommand{\NormalTok}[1]{{#1}}

    % Additional commands for more recent versions of Pandoc
    \newcommand{\ConstantTok}[1]{\textcolor[rgb]{0.53,0.00,0.00}{{#1}}}
    \newcommand{\SpecialCharTok}[1]{\textcolor[rgb]{0.25,0.44,0.63}{{#1}}}
    \newcommand{\VerbatimStringTok}[1]{\textcolor[rgb]{0.25,0.44,0.63}{{#1}}}
    \newcommand{\SpecialStringTok}[1]{\textcolor[rgb]{0.73,0.40,0.53}{{#1}}}
    \newcommand{\ImportTok}[1]{{#1}}
    \newcommand{\DocumentationTok}[1]{\textcolor[rgb]{0.73,0.13,0.13}{\textit{{#1}}}}
    \newcommand{\AnnotationTok}[1]{\textcolor[rgb]{0.38,0.63,0.69}{\textbf{\textit{{#1}}}}}
    \newcommand{\CommentVarTok}[1]{\textcolor[rgb]{0.38,0.63,0.69}{\textbf{\textit{{#1}}}}}
    \newcommand{\VariableTok}[1]{\textcolor[rgb]{0.10,0.09,0.49}{{#1}}}
    \newcommand{\ControlFlowTok}[1]{\textcolor[rgb]{0.00,0.44,0.13}{\textbf{{#1}}}}
    \newcommand{\OperatorTok}[1]{\textcolor[rgb]{0.40,0.40,0.40}{{#1}}}
    \newcommand{\BuiltInTok}[1]{{#1}}
    \newcommand{\ExtensionTok}[1]{{#1}}
    \newcommand{\PreprocessorTok}[1]{\textcolor[rgb]{0.74,0.48,0.00}{{#1}}}
    \newcommand{\AttributeTok}[1]{\textcolor[rgb]{0.49,0.56,0.16}{{#1}}}
    \newcommand{\InformationTok}[1]{\textcolor[rgb]{0.38,0.63,0.69}{\textbf{\textit{{#1}}}}}
    \newcommand{\WarningTok}[1]{\textcolor[rgb]{0.38,0.63,0.69}{\textbf{\textit{{#1}}}}}


    % Define a nice break command that doesn't care if a line doesn't already
    % exist.
    \def\br{\hspace*{\fill} \\* }
    % Math Jax compatability definitions
    \def\gt{>}
    \def\lt{<}
    % Document parameters
    \title{Untitled1}




    % Pygments definitions

\makeatletter
\def\PY@reset{\let\PY@it=\relax \let\PY@bf=\relax%
    \let\PY@ul=\relax \let\PY@tc=\relax%
    \let\PY@bc=\relax \let\PY@ff=\relax}
\def\PY@tok#1{\csname PY@tok@#1\endcsname}
\def\PY@toks#1+{\ifx\relax#1\empty\else%
    \PY@tok{#1}\expandafter\PY@toks\fi}
\def\PY@do#1{\PY@bc{\PY@tc{\PY@ul{%
    \PY@it{\PY@bf{\PY@ff{#1}}}}}}}
\def\PY#1#2{\PY@reset\PY@toks#1+\relax+\PY@do{#2}}

\expandafter\def\csname PY@tok@w\endcsname{\def\PY@tc##1{\textcolor[rgb]{0.73,0.73,0.73}{##1}}}
\expandafter\def\csname PY@tok@c\endcsname{\let\PY@it=\textit\def\PY@tc##1{\textcolor[rgb]{0.25,0.50,0.50}{##1}}}
\expandafter\def\csname PY@tok@cp\endcsname{\def\PY@tc##1{\textcolor[rgb]{0.74,0.48,0.00}{##1}}}
\expandafter\def\csname PY@tok@k\endcsname{\let\PY@bf=\textbf\def\PY@tc##1{\textcolor[rgb]{0.00,0.50,0.00}{##1}}}
\expandafter\def\csname PY@tok@kp\endcsname{\def\PY@tc##1{\textcolor[rgb]{0.00,0.50,0.00}{##1}}}
\expandafter\def\csname PY@tok@kt\endcsname{\def\PY@tc##1{\textcolor[rgb]{0.69,0.00,0.25}{##1}}}
\expandafter\def\csname PY@tok@o\endcsname{\def\PY@tc##1{\textcolor[rgb]{0.40,0.40,0.40}{##1}}}
\expandafter\def\csname PY@tok@ow\endcsname{\let\PY@bf=\textbf\def\PY@tc##1{\textcolor[rgb]{0.67,0.13,1.00}{##1}}}
\expandafter\def\csname PY@tok@nb\endcsname{\def\PY@tc##1{\textcolor[rgb]{0.00,0.50,0.00}{##1}}}
\expandafter\def\csname PY@tok@nf\endcsname{\def\PY@tc##1{\textcolor[rgb]{0.00,0.00,1.00}{##1}}}
\expandafter\def\csname PY@tok@nc\endcsname{\let\PY@bf=\textbf\def\PY@tc##1{\textcolor[rgb]{0.00,0.00,1.00}{##1}}}
\expandafter\def\csname PY@tok@nn\endcsname{\let\PY@bf=\textbf\def\PY@tc##1{\textcolor[rgb]{0.00,0.00,1.00}{##1}}}
\expandafter\def\csname PY@tok@ne\endcsname{\let\PY@bf=\textbf\def\PY@tc##1{\textcolor[rgb]{0.82,0.25,0.23}{##1}}}
\expandafter\def\csname PY@tok@nv\endcsname{\def\PY@tc##1{\textcolor[rgb]{0.10,0.09,0.49}{##1}}}
\expandafter\def\csname PY@tok@no\endcsname{\def\PY@tc##1{\textcolor[rgb]{0.53,0.00,0.00}{##1}}}
\expandafter\def\csname PY@tok@nl\endcsname{\def\PY@tc##1{\textcolor[rgb]{0.63,0.63,0.00}{##1}}}
\expandafter\def\csname PY@tok@ni\endcsname{\let\PY@bf=\textbf\def\PY@tc##1{\textcolor[rgb]{0.60,0.60,0.60}{##1}}}
\expandafter\def\csname PY@tok@na\endcsname{\def\PY@tc##1{\textcolor[rgb]{0.49,0.56,0.16}{##1}}}
\expandafter\def\csname PY@tok@nt\endcsname{\let\PY@bf=\textbf\def\PY@tc##1{\textcolor[rgb]{0.00,0.50,0.00}{##1}}}
\expandafter\def\csname PY@tok@nd\endcsname{\def\PY@tc##1{\textcolor[rgb]{0.67,0.13,1.00}{##1}}}
\expandafter\def\csname PY@tok@s\endcsname{\def\PY@tc##1{\textcolor[rgb]{0.73,0.13,0.13}{##1}}}
\expandafter\def\csname PY@tok@sd\endcsname{\let\PY@it=\textit\def\PY@tc##1{\textcolor[rgb]{0.73,0.13,0.13}{##1}}}
\expandafter\def\csname PY@tok@si\endcsname{\let\PY@bf=\textbf\def\PY@tc##1{\textcolor[rgb]{0.73,0.40,0.53}{##1}}}
\expandafter\def\csname PY@tok@se\endcsname{\let\PY@bf=\textbf\def\PY@tc##1{\textcolor[rgb]{0.73,0.40,0.13}{##1}}}
\expandafter\def\csname PY@tok@sr\endcsname{\def\PY@tc##1{\textcolor[rgb]{0.73,0.40,0.53}{##1}}}
\expandafter\def\csname PY@tok@ss\endcsname{\def\PY@tc##1{\textcolor[rgb]{0.10,0.09,0.49}{##1}}}
\expandafter\def\csname PY@tok@sx\endcsname{\def\PY@tc##1{\textcolor[rgb]{0.00,0.50,0.00}{##1}}}
\expandafter\def\csname PY@tok@m\endcsname{\def\PY@tc##1{\textcolor[rgb]{0.40,0.40,0.40}{##1}}}
\expandafter\def\csname PY@tok@gh\endcsname{\let\PY@bf=\textbf\def\PY@tc##1{\textcolor[rgb]{0.00,0.00,0.50}{##1}}}
\expandafter\def\csname PY@tok@gu\endcsname{\let\PY@bf=\textbf\def\PY@tc##1{\textcolor[rgb]{0.50,0.00,0.50}{##1}}}
\expandafter\def\csname PY@tok@gd\endcsname{\def\PY@tc##1{\textcolor[rgb]{0.63,0.00,0.00}{##1}}}
\expandafter\def\csname PY@tok@gi\endcsname{\def\PY@tc##1{\textcolor[rgb]{0.00,0.63,0.00}{##1}}}
\expandafter\def\csname PY@tok@gr\endcsname{\def\PY@tc##1{\textcolor[rgb]{1.00,0.00,0.00}{##1}}}
\expandafter\def\csname PY@tok@ge\endcsname{\let\PY@it=\textit}
\expandafter\def\csname PY@tok@gs\endcsname{\let\PY@bf=\textbf}
\expandafter\def\csname PY@tok@gp\endcsname{\let\PY@bf=\textbf\def\PY@tc##1{\textcolor[rgb]{0.00,0.00,0.50}{##1}}}
\expandafter\def\csname PY@tok@go\endcsname{\def\PY@tc##1{\textcolor[rgb]{0.53,0.53,0.53}{##1}}}
\expandafter\def\csname PY@tok@gt\endcsname{\def\PY@tc##1{\textcolor[rgb]{0.00,0.27,0.87}{##1}}}
\expandafter\def\csname PY@tok@err\endcsname{\def\PY@bc##1{\setlength{\fboxsep}{0pt}\fcolorbox[rgb]{1.00,0.00,0.00}{1,1,1}{\strut ##1}}}
\expandafter\def\csname PY@tok@kc\endcsname{\let\PY@bf=\textbf\def\PY@tc##1{\textcolor[rgb]{0.00,0.50,0.00}{##1}}}
\expandafter\def\csname PY@tok@kd\endcsname{\let\PY@bf=\textbf\def\PY@tc##1{\textcolor[rgb]{0.00,0.50,0.00}{##1}}}
\expandafter\def\csname PY@tok@kn\endcsname{\let\PY@bf=\textbf\def\PY@tc##1{\textcolor[rgb]{0.00,0.50,0.00}{##1}}}
\expandafter\def\csname PY@tok@kr\endcsname{\let\PY@bf=\textbf\def\PY@tc##1{\textcolor[rgb]{0.00,0.50,0.00}{##1}}}
\expandafter\def\csname PY@tok@bp\endcsname{\def\PY@tc##1{\textcolor[rgb]{0.00,0.50,0.00}{##1}}}
\expandafter\def\csname PY@tok@fm\endcsname{\def\PY@tc##1{\textcolor[rgb]{0.00,0.00,1.00}{##1}}}
\expandafter\def\csname PY@tok@vc\endcsname{\def\PY@tc##1{\textcolor[rgb]{0.10,0.09,0.49}{##1}}}
\expandafter\def\csname PY@tok@vg\endcsname{\def\PY@tc##1{\textcolor[rgb]{0.10,0.09,0.49}{##1}}}
\expandafter\def\csname PY@tok@vi\endcsname{\def\PY@tc##1{\textcolor[rgb]{0.10,0.09,0.49}{##1}}}
\expandafter\def\csname PY@tok@vm\endcsname{\def\PY@tc##1{\textcolor[rgb]{0.10,0.09,0.49}{##1}}}
\expandafter\def\csname PY@tok@sa\endcsname{\def\PY@tc##1{\textcolor[rgb]{0.73,0.13,0.13}{##1}}}
\expandafter\def\csname PY@tok@sb\endcsname{\def\PY@tc##1{\textcolor[rgb]{0.73,0.13,0.13}{##1}}}
\expandafter\def\csname PY@tok@sc\endcsname{\def\PY@tc##1{\textcolor[rgb]{0.73,0.13,0.13}{##1}}}
\expandafter\def\csname PY@tok@dl\endcsname{\def\PY@tc##1{\textcolor[rgb]{0.73,0.13,0.13}{##1}}}
\expandafter\def\csname PY@tok@s2\endcsname{\def\PY@tc##1{\textcolor[rgb]{0.73,0.13,0.13}{##1}}}
\expandafter\def\csname PY@tok@sh\endcsname{\def\PY@tc##1{\textcolor[rgb]{0.73,0.13,0.13}{##1}}}
\expandafter\def\csname PY@tok@s1\endcsname{\def\PY@tc##1{\textcolor[rgb]{0.73,0.13,0.13}{##1}}}
\expandafter\def\csname PY@tok@mb\endcsname{\def\PY@tc##1{\textcolor[rgb]{0.40,0.40,0.40}{##1}}}
\expandafter\def\csname PY@tok@mf\endcsname{\def\PY@tc##1{\textcolor[rgb]{0.40,0.40,0.40}{##1}}}
\expandafter\def\csname PY@tok@mh\endcsname{\def\PY@tc##1{\textcolor[rgb]{0.40,0.40,0.40}{##1}}}
\expandafter\def\csname PY@tok@mi\endcsname{\def\PY@tc##1{\textcolor[rgb]{0.40,0.40,0.40}{##1}}}
\expandafter\def\csname PY@tok@il\endcsname{\def\PY@tc##1{\textcolor[rgb]{0.40,0.40,0.40}{##1}}}
\expandafter\def\csname PY@tok@mo\endcsname{\def\PY@tc##1{\textcolor[rgb]{0.40,0.40,0.40}{##1}}}
\expandafter\def\csname PY@tok@ch\endcsname{\let\PY@it=\textit\def\PY@tc##1{\textcolor[rgb]{0.25,0.50,0.50}{##1}}}
\expandafter\def\csname PY@tok@cm\endcsname{\let\PY@it=\textit\def\PY@tc##1{\textcolor[rgb]{0.25,0.50,0.50}{##1}}}
\expandafter\def\csname PY@tok@cpf\endcsname{\let\PY@it=\textit\def\PY@tc##1{\textcolor[rgb]{0.25,0.50,0.50}{##1}}}
\expandafter\def\csname PY@tok@c1\endcsname{\let\PY@it=\textit\def\PY@tc##1{\textcolor[rgb]{0.25,0.50,0.50}{##1}}}
\expandafter\def\csname PY@tok@cs\endcsname{\let\PY@it=\textit\def\PY@tc##1{\textcolor[rgb]{0.25,0.50,0.50}{##1}}}

\def\PYZbs{\char`\\}
\def\PYZus{\char`\_}
\def\PYZob{\char`\{}
\def\PYZcb{\char`\}}
\def\PYZca{\char`\^}
\def\PYZam{\char`\&}
\def\PYZlt{\char`\<}
\def\PYZgt{\char`\>}
\def\PYZsh{\char`\#}
\def\PYZpc{\char`\%}
\def\PYZdl{\char`\$}
\def\PYZhy{\char`\-}
\def\PYZsq{\char`\'}
\def\PYZdq{\char`\"}
\def\PYZti{\char`\~}
% for compatibility with earlier versions
\def\PYZat{@}
\def\PYZlb{[}
\def\PYZrb{]}
\makeatother


    % Exact colors from NB
    \definecolor{incolor}{rgb}{0.0, 0.0, 0.5}
    \definecolor{outcolor}{rgb}{0.545, 0.0, 0.0}




    % Prevent overflowing lines due to hard-to-break entities
    \sloppy
    % Setup hyperref package
    \hypersetup{
      breaklinks=true,  % so long urls are correctly broken across lines
      colorlinks=true,
      urlcolor=urlcolor,
      linkcolor=linkcolor,
      citecolor=citecolor,
      }
    % Slightly bigger margins than the latex defaults

    \geometry{verbose,tmargin=1in,bmargin=1in,lmargin=1in,rmargin=1in}
\usepackage{geometry} % Required to change the page size to A4
\geometry{a4paper} % Set the page size to be A4 as opposed to the default US Letter
\usepackage{graphicx} % Required for including pictures


\usepackage[T1]{fontenc}

\usepackage{float} % Allows putting an [H] in \begin{figure} to specify the exact location of the figure
\usepackage{wrapfig} % Allows in-line images such as the example fish picture

\usepackage{lipsum} % Used for inserting dummy 'Lorem ipsum' text into the template

\linespread{1.2} % Line spacing

%\setlength\parindent{0pt} % Uncomment to remove all indentation from paragraphs

\graphicspath{{Pictures/}} % Specifies the directory where pictures are stored


\renewcommand\refname{Bibliografía}


\begin{document}

%----------------------------------------------------------------------------------------
%	TITLE PAGE
%----------------------------------------------------------------------------------------

\begin{titlepage}

\newcommand{\HRule}{\rule{\linewidth}{0.5mm}} % Defines a new command for the horizontal lines, change thickness here

\center % Center everything on the page

\textsc{\LARGE Universidad de Sonora}\\[1.5cm] % Name of your university/college
\textsc{\Large Reconocimiento de Patrones}\\[0.5cm] % Major heading such as course name

\HRule \\[0.4cm]
{ \huge \bfseries Sistemas de recomendacion: prediccion de preferencias}\\[0.4cm] % Title of your document
\HRule \\[1.5cm]

\begin{minipage}{0.4\textwidth}
\begin{flushleft} \large
\emph{Autor:}\\
Rafael Castillo % Your name
\end{flushleft}
\end{minipage}
~
\begin{minipage}{0.4\textwidth}
\begin{flushright} \large
\emph{Profesor:} \\
Dr. Ramón Soto de la Crúz % Supervisor's Name
\end{flushright}
\end{minipage}\\[4cm]

{\large \today}\\[3cm]

\vfill

\end{titlepage}
\tableofcontents

\newpage


\section{Resumen} 

Los sistemas de recomendación son métodos cuya importancia cada vez es mayor en
los sistemas que interactúan con usuarios. En este proyecto de implementaron
cuatro método de recomendación diferentes aplicados a la misma base de datos con
el fin de analizar las ventajas y desventajas de cada uno de ellos. En este
reporte se introduce el problema de recomendación, los factores que influyen en
el diseño de un sistema de recomendación, así como el tipo de soluciones que
existen para estos problemas. Se hace una exploración de diversos métodos de
predicción de calificaciones, incluyendo recomendación basada en contenido,
filtrado colaborativo basado en memoria y filtrado colaborativo basado en
modelos. Finalmente se propone una solución al problema del arranque en frió. Se
concluye que el metodo de factorización SVD tiene muchas ventajas sobre los
otros metodos pero que cada uno tiene un nicho que llenar.

\section{Introducción}

Un sistema de recomendación es un sistema que ayuda a un usuario encontrar algo
cuando el proceso de búsqueda es desafiante por la cantidad de opciones
existentes \cite{su2009survey}. La recomendación ha existido en la naturaleza y
en la humanidad por milenios. Cuando las hormigas buscan comida, primero todas
exploran el espacio. Si una hormiga encuentra comida, deja una feromona en esa
ubicación, lo que lleva a otras hormigas a ir a ese lugar. Estas hormigas están
basándose en la experiencia de otra para tomar decisiones, una recomendación.

Existen ejemplos de sistemas de recomendación desde antes de que los humanos
desarrollaran la escritura o el habla. Si un hombre prehistórico veía una planta
desconocida, tenia dos opciones: Si era un hombre que no conocía los sistemas de
recomendación, se arriesgaba y comía la planta; si era un hombre que sí sabía
sobre sistemas de recomendaciones, esperaba a que un cavernícola menos dotado
comiera la planta, si el cavernícola parecía disfrutarla, el también la comía,
pero si el otro caía al suelo dolorido, entonces su vida fue salvada gracias a
la recomendación del otro.

Asi, podemos ver que el concepto de la recomendación ha existido por mucho
tiempo. A través de toda nuestra historia hemos encontrado formas de poder
evaluar articulos sin tener que consumirlos nosotros mismos. Tradicionalmente
aprovechabamos a otros humanos que estaban dispuestos a consumir los articulos
que no queriamos (individuos tradicionalmente llamados "criticos"), pero en un
mundo donde la automatización esta por doquier, hemos desarrollado tecnicas que
nos permiten generar recomendaciones a partir de datos recolectados de usuarios,
aprovechando la llamada "inteligencia collectiva" de muchos para aproximar los
gustos de un individuo

Los sistemas de recomendación tienen un alto valor. Las recomendaciones pueden
mostrarse en varias interfaces, amplificando su eficacia. Por ejemplo, las
recomendaciones pueden tener interfaces obvias como listas de sugerencia, pero
también pueden existir interfaces mas sutiles, como filtros de correo
electrónico o sistemas de evaluación de candidatos de empleo.

Por la naturaleza de muchos de estos metodos, como muchos otros metodos de
aprendizaje, funcionan mucho mejor cuando la cantidad de datos es mucho mayor.
Por esto, muchas empresas muestan interes en aprovechar la gran cantidad de
datos que recolectan para crear sistemas de recomendación, ya que son un recurso
que tiene un gran impacto enriquecedor en las experiencias de usuario.

\subsection{Familias de métodos de recomendación}

Hay muchas formas de lograr hacer recomendaciones automáticas, algunas mas
sofisticadas que otras. Existen los métodos sencillos no personalizados, que se
basan simplemente en métricas como popularidad. Otro método es usar la llamada
asociación de productos, es decir recomendar a un usuario artículos similares a
uno que este viendo. Estos métodos no no son de mucho interés, pues las
recomendaciones no personalizadas no son de tanto valor. Por esta razón nos
centraremos en los otros dos métodos.

Un categoría de métodos capaces de personalizar las recomendaciones a un usuario
la de los métodos basados en contenido. Aquí aprovechamos conocimiento previo
sobre los artículos (características de estos) para aproximar los gustos de un
usuario a partir de las calificaciones que ya ha impartido. Este método también
es conocido como filtrado basado en contenido y esta estrechamente relacionado
al problema mas general de filtrado de información.

La ultima familia de métodos a mencionar son los del filtrado colaborativo. La
idea del filtrado colaborativo es aprovechar las calificaciones de todos los
usuarios del sistema para poder predecir preferencias y dar recomendaciones. El
filtrado colaborativo además se separa en dos categorías: el filtrado que
utiliza todas las calificaciones en el sistema, conocido como filtrado basado en
memoria; y el filtrado que utiliza aprendizaje para poder predecir
calificaciones, conocido como filtrado basado en modelos.

Una ventaja del filtrado colaborativo es que tiene mas potencial de encontrar
correlaciones sorprendentes entre ciertos articulos muy distintos, algo que
dificilmente sucederia en un metodo basado en contenido. Esto ocurre porque los
metodos de filtrado colaborativo existen solamente basados en las preferencias
de los usuarios, no en preconcepciones sobre el contenido de los articulos,
llevando a recomendaciones basadas más en opinion colectiva que en prejuicios
sobre los articulos.

\subsection{Calificaciones} % Sub-section

La base para el filtrado colaborativo y los sistemas de recomendación en
general, una calificación es un indicador de la preferencia que tiene un usuario
hacia un articulo. Hablaremos de calificaciones en dos categorías, implícitas y
explicitas.

Las calificaciones explicitas son en las que se vienen a la cabeza cuando
pensamos en calificación. Una calificación se considera explicita cuando el
usuario explícitamente expresa su opinión sobre un articulo. Esto puede ser
desde una escala numérica como las muy comunes de 1 a 5 estrellas, hasta un
sistema de votos como el de Facebook o Reddit. Las escalas mas granulares tienen
la ventaja de ser mas expresivas, mientras que un voto binario es mas fácil para
el usuario, lo que lo hace útil para contenido mas efímero, como una noticia o
un tweet.

Una calificación implícita se extrae de comportamiento del usuario que no tiene
relación directa con su preferencia hacia un articulo. Hacer clic sobre un
enlace puede considerarse una calificación positiva hacia este. Ver un vídeo
hasta el final puede considerarse una calificación positiva, mientras de que
dejarlo de ver a la mitad es uno negativo. Estas calificaciones implícitas
pueden significar mucho y ser muy abundantes, dando buenas recomendaciones.

La variedad de fuentes de donde se pueden minar calificaciones es una de los
elementos que hacen que los sistemas de recomendación sean útiles en
aplicaciones donde anteriormente no se considerarían. Esto combinado con la
recolección masiva de datos le da a las grandes corporaciones nuevas y
emocionantes formas de llevar a cabo practicas horripilantes.


\section{Datos para el desarrollo de sistemas de recomendación}

Existen varios conjuntos de datos públicamente disponibles que se utilizan para
el desarrollo de sistemas de recomendación. Estos son los que se consideraron
para el desarrollo de los métodos.

\begin{description}
\item [MovieLens \cite{movielens}] es una base de datos de calificaciones de
  película. Se encuentra disponible en varios tamaños, pero aquí solo se tomara
  en cuenta la versión pequeña (100,000 calificaciones sobre 9,000 películas por
  700 usuarios).
\item [Book Crossings \cite{bookcrossing}] es un conjunto de datos de
  calificaciones de libros. Tiene 1 millón de calificaciones para 90,000
  usuarios en 270,000 libros, volviéndola la base de datos con calificaciones
  mas dispersas.
\item [Jester \cite{eigentaste}] es un conjunto de datos que incluye 5 millones
  de calificaciones por 100,000 usuarios en 150 chistes. Esta es la base de
  datos mas densa.
\end{description}

Estos son conjuntos de datos explicitamente generados para desarrollo de
sistemas recomendadores, como se mencionó antes, las calificaciones necesarias
para esta clase de sistemas pueden surgir de lugares inesperados. Por ejemplo,
se puede obtener una calificación a partir del tiempo tomado para leer un
articulo, o por analisis de sentimientos aplicado a un comentario dejado en un
video.

Para la mayoría de los métodos de filtrado colaborativo, necesitamos que los
datos tengan la forma de una matriz usuario-articulo (user-item matrix), que
tiene como filas cada articulo y como columnas a los usuarios. El único dataset
que viene ya en este formato es Jester, los demás se encuentran en un formato
mas parecido al de una base de datos relacional. En el apéndice \ref{subsec:a1},
se muestra un ejemplo para accesar a la base de datos \emph{movielens}

Para la prueba de los métodos que se presentaran, se utilizo la base de datos de
MovieLens, ya que viene en varios tamaños y tiene una densidad adecuada para las
pruebas que realizaremos.

\section{Métodos para predecir preferencias}

El componente mas crucial para el desarrollo de un sistema de recomendaciones es
el que predice las preferencias de un usuario por un articulo. A continuación se
presenta la descripción e implementación de diversos métodos para predecir
preferencias utilizados para el desarrollo de sistemas de recomendaciones.

\subsection{Recomendaciones basadas en contenido} % Major section

Los primeros métodos a revisar son los que se basan en características ya bien
conocidas sobre los artículos calificados \cite{contentbased}. Estos métodos son
también conocidos como métodos de filtrado de contenido ya que utilizan las
características del

La idea es que dado un vector con $N$ características $x^{(i)}$ y conociendo un
vector $\theta^(j)$, también de tamaño $N$ donde $\theta_k^{(j)}$ sea la
preferencia del usuario $j$ por la característica $k$, podemos predecir la
calificación del articulo $i$ esta dada por

$$(\theta^{(j)})^Tx^{(i)}$$

La ventaja de este método es que puramente se basa en contenido. No se necesita
tomar en cuenta las calificaciones de usuarios además de la del usuario activo.
Por esta razón este método tiene un bajo uso de recursos, lo cual lo hace
deseable para muchas aplicaciones.

La desventaja es la obvia: es necesario tener caracteristicas concretas de los
articulos previo a la aplicación del metodo. Muchas veces esto hace que el
metodo sea indeseable para aplicaciones donde hay una cantidad alta de
articulos, o en situaciones donde es dificil cuantificar calidades de los
articulos, como en un sistema de recomendación de arte. Pocas veces escuchas "me
gustan las pinturas azules" en lugar de "me gustan las pinturas que me hacen
sentir melancolico".

\subsubsection{Características de los artículos}

La primera parte que define un método basado en contenido es la fuente de donde
se obtienen las características de los artículos. La opción mas obvia es irse
por la ruta del conocimiento experto y asignar las características manualmente
para cada articulo. Este método puede sonar demasiado ingenuo pero puede ser
útil en muchos casos, por ejemplo para asignar características técnicas o
fácilmente cuantificables a un conjunto de artículos similares (por ejemplo, en
una tienda donde solo vendan un tipo de articulo).

También es factible crear interfaces para que los usuarios mismos determinen las
características de los artículos. Este método puede llevar a la generación de
una cantidad enorme de características, pero no es necesariamente confiable.
Muchos servicios utilizan esta opción en forma de etiquetas que un usuario puede
asignar a un articulo y que un ingeniero puede aprovechar después para
determinar características de este.

Una ultima opción a mencionar es la extracción automática de características de
un articulo. Por ejemplo, las palabras de un nota periodística pueden ser
característicos de este, como lo pueden ser las frecuencias mas importantes de
una canción, o la longitud de un vídeo.

\subsubsection{Gustos de los usuarios}

La pregunta ahora se vuelve como determinar los gustos del usuario $j$, dados
por $\theta^{(j)}$. Una forma posible es preguntarle estos datos al usuario.
Este enfoque se ha usado históricamente, por ejemplo en aplicaciones que le
piden a un usuario que especifique sus gustos o que presentan cuestionarios
acerca del contenido que quisiera ver.

Un enfoque tal vez mas interesante es usar las calificaciones previas del
usuario para aproximar $\theta^{(j)}$. Una opción simple es simplemente hacer
una suma de las características de los artículos $X_j$ que ya consumió el
usuario:

$$\theta^{(j)} = \sum_{i\in X_j}r_{i, j}x^{(i)}$$

También podemos modelar los gustos como un problema de regresión lineal donde
nuestra función de error esta dada por

$$\frac{1}{2}\sum_{i:r(i, j)=1} {\Big((\theta^{(j)})^Tx^{(i)}\Big)^2} +
\frac{\lambda}{2}\sum_{k=1}^n {(\theta^{(j)}_k)^2}$$

Si encontramos el $\theta^{(j)}$ que minimice este error, tendremos una
aproximación de los gustos del usuario, que podremos usar para predecir sus
calificaciones a artículos que no conoce aun.

En el anexo \ref{subsec:a2} se presentan una implementación del método de
vecinos mas cercanos para la base de datos de \emph{MovieLens}


\subsection{Vecinos mas cercanos}

El siguiente método que veremos es un método de filtrado colaborativo, es decir,
que no solo toma en cuenta las calificaciones del usuario para el cual estamos
prediciendo calificaciones, sino que tomamos en cuenta las preferencias de
nuestros usuarios.

El método utiliza una búsqueda de los k vecinos mas cercanos para encontrar una
vecindad del usuario para el cual queremos predecir preferencias. Una vez que
encontremos a los k vecinos mas cercanos $U_a$ y sus respectivas distancias,
podemos predecir la calificación del usuario $a$ en el articulo $i$ como

$$p_{a, i} = \frac {\sum\limits_{i\in U_a} w_{u, i}r_{ui}} {\sum\limits_{i\in U_a} w_{u, i}}$$

La idea aqui es encontrar una vecindad de usuarios que compartan gustos con el
usuario, gustos que usaremos despues para predecir sus calificaciones futuras.
De esta forma, aprovechamos la opinion colectiva de usuarios similares para
obtener una predicción decente de las preferencias de un individuo.

La predicción de preferencias usando este método produce resultados decentes y
el tiempo de computo para una calificación predicha es mínima. La desventaja
obvia del método es que requiere que toda la base de datos se mantenga en
memoria, por lo que este método puede volverse prohibitivo para conjuntos de
datos muy grandes si no se realizan trucos de optimización.

El método clásico para filtrado colaborativo usando vecinos cercanos surgió en
el grupo de investigación GroupLens, y esta descrito en detalle por Sarwar et al.
en \cite{itembased}.

En el anexo \ref{subsec:a3} se presenta un ejemplo del metodo de vecinos mas
cercanos para el filtrado colaborativo en la base de datos de MovieLens

\subsection{Filtrado colaborativo usando regresión}

En la sección de de recomendación basada en contenido, se hablo de un método de
aproximar $\theta^{(j)}$ usando regresión. También es claro ver que se puede
hacer un proceso similar para aproximar las características $x^{(i)}$ de un
articulo conociendo previamente los gustos  $\theta^{(j)}$ de un usuario. Estos
dos procesos de regresión pueden extenderse para calcular las características
$x^{(i)}$ para todas las $i$ resultando en una matriz $X$, así como una análoga
matriz $\Theta$. Aun mas allá, podemos aproximar estos dos dentro de la misma
regresión lineal. Llevando a cabo este proceso, efectivamente obtenemos una
descomposición de la matriz $R$ que nos permite aproximar todas las
calificaciones de nuestros usuarios simplemente calculando $\Theta^T X$.

Este método aprovecha todos los valores de la matriz de calificaciones para
predecir calificaciones faltantes, convirtiéndolo en un método de filtrado
colaborativo. Además, utiliza un modelo de aprendizaje en lugar de mantener
todas las calificaciones en memoria, volviéndolo un método de filtrado
colaborativo basado en modelo\cite{vucetic2005collaborative}.

Al utilizar este método, tenemos la ventaja de que una vez que se completa el
entrenamiento, el uso de memoria se reduce. Además de esto, obtenemos como
producto secundario una matriz $X$ aproximada. Podemos considerar esta matriz
$X$ una matriz de características arbitraria (aunque esta contenga mas que nada
características difíciles de interpretar para un humano), que podemos usar para
realizar operaciones que requieran de una matriz como esta, incluyendo uno que
veremos en una seccion futura.

Este y otros métodos basados en modelos que resultan en dos matrices de rango
menor, predeciblemente son conocidos como factorizacion de matrices de rank
menor (low rank matrix factorization).

En el anexo \ref{subsec:a4} se incluye una implementación de la descomposición
basada en regresión y como se utilizaría para obtener recomendaciones.

\subsubsection{Normalización de medias}

Al aplicar el método de regresión sobre nuestros datos, podemos darnos cuenta de
un detalle peculiar sobre su naturaleza: para un usuario nuevo, todas sus
calificaciones se aproximan como 0. Esta observación es parte del problema de
arranque en frió, que veremos una sección futura.

Una forma de remediar este problema es restar a cada calificación existente la
media de calificaciones para ese articulo en particular. Haciendo esto, una
calificación 0, en vez de tener un valor arbitrario, se vuelve la calificación
media de ese articulo. Esto ocasiona que para un usuario nuevo, se prediga una
calificación promedio para cada articulo, un resultado mas razonable.

\subsubsection{Descomposición por valores singulares}

SVD es una técnica de factorización matricial que busca reducir las
características de una matriz a un espacio vectorial $K < N$. SVD nos deja con
tres matrices tales que $R = U×S×V^T$ \cite{svd}. Lo interesante de esto es que la
factorización SVD es equivalente al proceso de regresión descrito arriba.
La diferencia radica en que existen algoritmos ya optimizados para calcular la
SVD de una matriz incluidas en paquetes de álgebra lineal populares, como
LAPACK.

El método SVD es de los mas importantes y mas usados para el desarrollo de
sistemas de recomendación. Del 2006 al 2009, Netflix llevo a cabo una
competencia para determinar el mejor algoritmo de filtrado colaborativo
\cite{svd}. En el 2009 finalmente se otorgo el premio a un equipo que había
desarrollado un método basado en la descomposición SVD, mostrando que al menos
para el caso de uso de Netflix, el SVD es el método mas óptimo.

En el anexo \ref{subsec:a5}, se incluye una aplicación de las recomendaciones
usando una factorización SVD.

\section{El problema de arranque en frío}
\label{clustering}

Todos los métodos de predicción que se vieron solo pueden dar buenas
recomendaciones si el usuario ya tiene calificaciones en el sistema.El problema
del arranque en frió radica en como presentarle recomendaciones a un usuario
nuevo.

A continuación proponemos un método como solución para el problema de arranque
en frió. El requerimiento para aplicar este método es solamente contar con
un conjunto de vectores de características de las películas. Estos pueden
surgir directa o indirectamente para los métodos basados en contenido o de
descomposición matricial. Incluso es posible obtener resultados aplicando este
método directamente a la matriz de calificaciones.

La idea general del método es agrupar los artículos del sistema según sus
características. Se debe buscar dividir los artículos de una forma que se logra
división representativa, pero que no en demasiados grupos, pues la idea es que
el usuario califique un articulo de cada grupo.

Para crear los grupos de artículos, se empleo el método de las K medias. No es
necesario el uso de un algoritmo de clustering mas sofisticado, ya que el
propósito es crear una separación mas o menos uniforme del espacio, no
encontrar agrupamientos lógicos de los artículos.

Una vez que dividimos los artículos, proponemos presentar como recomendaciones
el articulo mas relevante de cada cluster. Para este propósito, necesitamos
definir una medida de relevancia, inspirados por el TFIDF, proponemos la
siguiente métrica para la relevancia $t_i$ para el articulo $i$

$$t_i = \sum\limits_{j\in U}R_{i, j}log\big(1 + \frac{N}{n_i}\big)$$

Una vez que obtenemos las medidas de relevancia de los artículos, obtenemos el
articulo mas relevante de cada cluster, que es el que recomendaremos al usuario
inicialmente. Estas recomendaciones son una muestra significativa del espacio
gracias al clustering, pero gracias a la medida de relevancia tambien resulta en
articulos a los cuales el usuario probablemente ya ha sido expuesto antes,
permitiendo que el sistema tenga una idea inicial de su gusto decente.

Una ejemplo de implementación de este metodo puede encontrarse en el anexo
\ref{subsec:a6}.

\section{Nota sobre codigo fuente}

Todo el codigo relacionado a este proyecto se encuentra en linea en el siguiente
repositorio de GitHub: \url{https://github.com/eltrufas/Sistemas-recomendacion}.
Ahi tambien se encuentra el codigo fuente usado para generar este reporte y
algunos otros programas adicionales que se desarrollaron en relación a sistemas
de recomendación.

\section{Conclusión} % Major section



La variedad de enfoques que se pueden tomar para el desarrollo de sistemas de
recomendación hace que sea un área interesante y rica del reconocimiento de
patrones. Si se tiene un conocimiento solido del área, los sistemas de
recomendación pueden presentar aplicaciones en muchos dominios. La tabla
\ref{comparacion} muestra una comparación cualitativa de los distintos metodos
de recomendación revisados.


\begin{table}[t]
    \centering
    \begin{tabular}{|l|c|c|c|}
    \hline
    Metodo & Filtrado de contenido & Vecinos mas cercanos & Regresión/SVD\\
    \hline
    Facilidad de implementación & $\star$ & $\star\star\star$ & $\star\star$\\
    Flexibilidad & $\star$ & $\star\star\star$ & $\star\star\star$ \\
    Calidad de recomendaciones & $\star\star$ & $\star\star$ & $\star\star\star$ \\
    Uso de memoria & $\star\star\star$ & $-\star$ & $\star\star\star$ \\
    \hline
    \end{tabular}
    \caption{Comparación de metodos de predicción de preferencias}
    \label{comparacion}
    \end{table}
   
En general, la conclusión a la que se llego es que para sistemas modernos donde
se quieren recomendaciones de alta calidad con muchos usuarios, SVD es el metodo
a cual acudir. Esto se ve claramente en el campo de investigación sobre sistemas
recomendadores, donde muchos de los nuevos avances son relacionados a la
factorización SVD, incluyendo metodos que permiten calcular la descomposición de
forma distribuida.

Esto no significa que los otros metodos no tengan su lugar. Los metodos basados
en contenido son utiles para campos donde existe mucha información experta o
donde se puede minar conocimiento colectivo de otra forma. Los metodos de
vecinos mas cercanos son mas faciles de implementar y funcionan bien cuando la
proporción del sistema es adecuada.

Los métodos de recomendaciones basadas en contenido y de filtrado colaborativo
tienen un lugar importante en el desarrollo de aplicaciones modernas, en parte
impulsado por el boom de datos moderno. La recolección de datos masivos hace que
sea posible derivar datos de calificación de lugares sorprendentes, abriendo las
puertas de creatividad a sistemas de recomendación nuevos e interesantes.

\bibliographystyle{unsrt}
\addcontentsline{toc}{section}{Bibliografía}
\bibliography{references}


\appendix
\section{Anexos}

\subsection{Ejemplo de acceso a la base de datos \emph{Movielens}}
\label{subsec:a1}

    \begin{Verbatim}[commandchars=\\\{\}]
{\color{incolor}In [{\color{incolor}3}]:} \PY{k+kn}{import} \PY{n+nn}{numpy} \PY{k}{as} \PY{n+nn}{np}
        \PY{k+kn}{import} \PY{n+nn}{pandas} \PY{k}{as} \PY{n+nn}{pd}

        \PY{n}{df} \PY{o}{=} \PY{n}{pd}\PY{o}{.}\PY{n}{read\PYZus{}csv}\PY{p}{(}\PY{l+s+s1}{\PYZsq{}}\PY{l+s+s1}{datasets/ml\PYZhy{}latest\PYZhy{}small/ratings.csv}\PY{l+s+s1}{\PYZsq{}}\PY{p}{)}
        \PY{n}{df}\PY{o}{.}\PY{n}{columns} \PY{o}{=} \PY{p}{[}\PY{l+s+s1}{\PYZsq{}}\PY{l+s+s1}{user\PYZus{}id}\PY{l+s+s1}{\PYZsq{}}\PY{p}{,} \PY{l+s+s1}{\PYZsq{}}\PY{l+s+s1}{movie\PYZus{}id}\PY{l+s+s1}{\PYZsq{}}\PY{p}{,} \PY{l+s+s1}{\PYZsq{}}\PY{l+s+s1}{rating}\PY{l+s+s1}{\PYZsq{}}\PY{p}{,} \PY{l+s+s1}{\PYZsq{}}\PY{l+s+s1}{timestamp}\PY{l+s+s1}{\PYZsq{}}\PY{p}{]}

        \PY{n}{df}\PY{o}{.}\PY{n}{head}\PY{p}{(}\PY{p}{)}
\end{Verbatim}

\begin{Verbatim}[commandchars=\\\{\}]
{\color{outcolor}Out[{\color{outcolor}3}]:}    user\_id  movie\_id  rating   timestamp
        0        1        31     2.5  1260759144
        1        1      1029     3.0  1260759179
        2        1      1061     3.0  1260759182
        3        1      1129     2.0  1260759185
        4        1      1172     4.0  1260759205
\end{Verbatim}

Podemos usar pandas para preparar los datos a el formato que necesitamos

\begin{Verbatim}[commandchars=\\\{\}]
{\color{incolor}In [{\color{incolor}4}]:} \PY{n}{ui\PYZus{}df} \PY{o}{=} \PY{n}{df}\PY{o}{.}\PY{n}{pivot}\PY{p}{(}\PY{n}{index}\PY{o}{=}\PY{l+s+s1}{\PYZsq{}}\PY{l+s+s1}{movie\PYZus{}id}\PY{l+s+s1}{\PYZsq{}}\PY{p}{,} \PY{n}{columns}\PY{o}{=}\PY{l+s+s1}{\PYZsq{}}\PY{l+s+s1}{user\PYZus{}id}\PY{l+s+s1}{\PYZsq{}}\PY{p}{,} \PY{n}{values}\PY{o}{=}\PY{l+s+s1}{\PYZsq{}}\PY{l+s+s1}{rating}\PY{l+s+s1}{\PYZsq{}}\PY{p}{)}

        \PY{n}{ui\PYZus{}df}\PY{o}{.}\PY{n}{head}\PY{p}{(}\PY{p}{)}
\end{Verbatim}


\begin{Verbatim}[commandchars=\\\{\}]
{\color{outcolor}Out[{\color{outcolor}4}]:} user\_id   1    2    3    4    5    6    7    8    9    10  {\ldots}   662  663
        movie\_id                                                   {\ldots}
        1         NaN  NaN  NaN  NaN  NaN  NaN  3.0  NaN  4.0  NaN {\ldots}   NaN  4.0
        2         NaN  NaN  NaN  NaN  NaN  NaN  NaN  NaN  NaN  NaN {\ldots}   5.0  NaN
        3         NaN  NaN  NaN  NaN  4.0  NaN  NaN  NaN  NaN  NaN {\ldots}   NaN  NaN
        4         NaN  NaN  NaN  NaN  NaN  NaN  NaN  NaN  NaN  NaN {\ldots}   NaN  NaN
        5         NaN  NaN  NaN  NaN  NaN  NaN  NaN  NaN  NaN  NaN {\ldots}   NaN  NaN


        [5 rows x 671 columns]
\end{Verbatim}

\subsection{Filtrado de contenido para \emph{MovieLens}}
\label{subsec:a2}

MovieLens incluye una base de datos de etiquetas así como la relevancia de cada etiqueta a cada película. Podemos usar estas relevancias para crear nuestra matriz de características para las películas del conjunto. Hagamos esto y veamos cuales son las etiquetas mas relevantes para Toy Story.

    \begin{Verbatim}[commandchars=\\\{\}]
{\color{incolor}In [{\color{incolor}1}]:} \PY{k+kn}{import} \PY{n+nn}{pandas} \PY{k}{as} \PY{n+nn}{pd}
        \PY{k+kn}{import} \PY{n+nn}{numpy} \PY{k}{as} \PY{n+nn}{np}

        \PY{n}{movies} \PY{o}{=} \PY{n}{pd}\PY{o}{.}\PY{n}{read\PYZus{}csv}\PY{p}{(}\PY{l+s+s1}{\PYZsq{}}\PY{l+s+s1}{datasets/ml\PYZhy{}20m/movies.csv}\PY{l+s+s1}{\PYZsq{}}\PY{p}{,} \PY{n}{index\PYZus{}col}\PY{o}{=}\PY{l+m+mi}{0}\PY{p}{)}

        \PY{n}{genome\PYZus{}tags} \PY{o}{=} \PY{n}{pd}\PY{o}{.}\PY{n}{read\PYZus{}csv}\PY{p}{(}\PY{l+s+s1}{\PYZsq{}}\PY{l+s+s1}{datasets/ml\PYZhy{}20m/genome\PYZhy{}tags.csv}\PY{l+s+s1}{\PYZsq{}}\PY{p}{,} \PY{n}{index\PYZus{}col}\PY{o}{=}\PY{l+m+mi}{0}\PY{p}{)}\PY{o}{.}\PY{n}{tag}
        \PY{n}{genome\PYZus{}scores} \PY{o}{=} \PY{n}{pd}\PY{o}{.}\PY{n}{read\PYZus{}csv}\PY{p}{(}\PY{l+s+s1}{\PYZsq{}}\PY{l+s+s1}{datasets/ml\PYZhy{}20m/genome\PYZhy{}scores.csv}\PY{l+s+s1}{\PYZsq{}}\PY{p}{)}
        \PY{n}{genome\PYZus{}scores}\PY{o}{.}\PY{n}{columns} \PY{o}{=} \PY{p}{[}\PY{l+s+s1}{\PYZsq{}}\PY{l+s+s1}{movie\PYZus{}id}\PY{l+s+s1}{\PYZsq{}}\PY{p}{,} \PY{l+s+s1}{\PYZsq{}}\PY{l+s+s1}{tag\PYZus{}id}\PY{l+s+s1}{\PYZsq{}}\PY{p}{,} \PY{l+s+s1}{\PYZsq{}}\PY{l+s+s1}{relevance}\PY{l+s+s1}{\PYZsq{}}\PY{p}{]}
        \PY{n}{genome\PYZus{}matrix} \PY{o}{=} \PY{n}{genome\PYZus{}scores}\PY{o}{.}\PY{n}{pivot}\PY{p}{(}\PY{n}{columns}\PY{o}{=}\PY{l+s+s1}{\PYZsq{}}\PY{l+s+s1}{tag\PYZus{}id}\PY{l+s+s1}{\PYZsq{}}\PY{p}{,} \PY{n}{index}\PY{o}{=}\PY{l+s+s1}{\PYZsq{}}\PY{l+s+s1}{movie\PYZus{}id}\PY{l+s+s1}{\PYZsq{}}\PY{p}{,}
                                            \PY{n}{values}\PY{o}{=}\PY{l+s+s1}{\PYZsq{}}\PY{l+s+s1}{relevance}\PY{l+s+s1}{\PYZsq{}}\PY{p}{)}
        \PY{n}{genome\PYZus{}matrix}\PY{o}{.}\PY{n}{columns} \PY{o}{=} \PY{p}{[}\PY{n}{genome\PYZus{}tags}\PY{p}{[}\PY{n+nb}{id}\PY{p}{]} \PY{k}{for} \PY{n+nb}{id} \PY{o+ow}{in} \PY{n}{genome\PYZus{}matrix}\PY{o}{.}\PY{n}{columns}\PY{p}{]}

        \PY{c+c1}{\PYZsh{} reducimos las peliculas a las que tienen etiquetas}
        \PY{n}{movies} \PY{o}{=} \PY{n}{movies}\PY{o}{.}\PY{n}{loc}\PY{p}{[}\PY{n}{genome\PYZus{}matrix}\PY{o}{.}\PY{n}{index}\PY{p}{]}

        \PY{c+c1}{\PYZsh{} normalizamos el genoma}
        \PY{n}{genome\PYZus{}matrix} \PY{o}{=} \PY{n}{genome\PYZus{}matrix} \PY{o}{*} \PY{l+m+mi}{2} \PY{o}{\PYZhy{}} \PY{l+m+mi}{1}

        \PY{n}{genome\PYZus{}matrix}\PY{o}{.}\PY{n}{loc}\PY{p}{[}\PY{l+m+mi}{1}\PY{p}{]}\PY{o}{.}\PY{n}{sort\PYZus{}values}\PY{p}{(}\PY{n}{ascending}\PY{o}{=}\PY{k+kc}{False}\PY{p}{)}\PY{o}{.}\PY{n}{head}\PY{p}{(}\PY{l+m+mi}{10}\PY{p}{)}
\end{Verbatim}


\begin{Verbatim}[commandchars=\\\{\}]
{\color{outcolor}Out[{\color{outcolor}1}]:} toys                  0.9985
        computer animation    0.9970
        pixar animation       0.9920
        kids and family       0.9815
        animation             0.9715
        kids                  0.9585
        pixar                 0.9335
        children              0.9285
        cartoon               0.9130
        imdb top 250          0.8840
        Name: 1, dtype: float64
\end{Verbatim}

Ahora vamos a cargar el archivo de calificaciones y seleccionar las
calificaciones de un usuario aleatorio.

    \begin{Verbatim}[commandchars=\\\{\}]
{\color{incolor}In [{\color{incolor}2}]:} \PY{n}{ratings} \PY{o}{=} \PY{n}{pd}\PY{o}{.}\PY{n}{read\PYZus{}csv}\PY{p}{(}\PY{l+s+s1}{\PYZsq{}}\PY{l+s+s1}{datasets/ml\PYZhy{}20m/ratings.csv}\PY{l+s+s1}{\PYZsq{}}\PY{p}{)}
        \PY{n}{ratings}\PY{o}{.}\PY{n}{columns} \PY{o}{=} \PY{p}{[}\PY{l+s+s1}{\PYZsq{}}\PY{l+s+s1}{user\PYZus{}id}\PY{l+s+s1}{\PYZsq{}}\PY{p}{,} \PY{l+s+s1}{\PYZsq{}}\PY{l+s+s1}{movie\PYZus{}id}\PY{l+s+s1}{\PYZsq{}}\PY{p}{,} \PY{l+s+s1}{\PYZsq{}}\PY{l+s+s1}{rating}\PY{l+s+s1}{\PYZsq{}}\PY{p}{,} \PY{l+s+s1}{\PYZsq{}}\PY{l+s+s1}{timestamp}\PY{l+s+s1}{\PYZsq{}}\PY{p}{]}
        \PY{c+c1}{\PYZsh{} reducimos a solo las peliculas con etiquetas}
        \PY{n}{ratings} \PY{o}{=} \PY{n}{ratings}\PY{o}{.}\PY{n}{loc}\PY{p}{[}\PY{n}{ratings}\PY{o}{.}\PY{n}{movie\PYZus{}id}\PY{o}{.}\PY{n}{isin}\PY{p}{(}\PY{n}{movies}\PY{o}{.}\PY{n}{index}\PY{p}{)}\PY{p}{,} \PY{p}{:}\PY{p}{]}
\end{Verbatim}


    \begin{Verbatim}[commandchars=\\\{\}]
{\color{incolor}In [{\color{incolor}3}]:} \PY{n}{user} \PY{o}{=} \PY{n}{np}\PY{o}{.}\PY{n}{random}\PY{o}{.}\PY{n}{choice}\PY{p}{(}\PY{n}{ratings}\PY{o}{.}\PY{n}{user\PYZus{}id}\PY{o}{.}\PY{n}{unique}\PY{p}{(}\PY{p}{)}\PY{p}{)}

        \PY{n}{user\PYZus{}ratings} \PY{o}{=} \PY{n}{ratings}\PY{o}{.}\PY{n}{loc}\PY{p}{[}\PY{n}{ratings}\PY{o}{.}\PY{n}{user\PYZus{}id} \PY{o}{==} \PY{n}{user}\PY{p}{]}
        \PY{n}{user\PYZus{}ratings} \PY{o}{=} \PY{n}{user\PYZus{}ratings}\PY{o}{.}\PY{n}{assign}\PY{p}{(}\PY{n}{title}\PY{o}{=}\PY{p}{[}\PY{n}{movies}\PY{o}{.}\PY{n}{loc}\PY{p}{[}\PY{n+nb}{id}\PY{p}{]}\PY{o}{.}\PY{n}{title}
                                                  \PY{k}{for} \PY{n+nb}{id} \PY{o+ow}{in} \PY{n}{user\PYZus{}ratings}\PY{o}{.}\PY{n}{movie\PYZus{}id}\PY{p}{]}\PY{p}{)}

        \PY{n+nb}{print}\PY{p}{(}\PY{n}{user\PYZus{}ratings}\PY{o}{.}\PY{n}{shape}\PY{p}{)}
        \PY{n}{user\PYZus{}ratings}\PY{o}{.}\PY{n}{sort\PYZus{}values}\PY{p}{(}\PY{n}{by}\PY{o}{=}\PY{l+s+s1}{\PYZsq{}}\PY{l+s+s1}{rating}\PY{l+s+s1}{\PYZsq{}}\PY{p}{,} \PY{n}{ascending}\PY{o}{=}\PY{k+kc}{False}\PY{p}{)}\PY{o}{.}\PY{n}{head}\PY{p}{(}\PY{l+m+mi}{20}\PY{p}{)}
\end{Verbatim}


    \begin{Verbatim}[commandchars=\\\{\}]
(82, 5)

    \end{Verbatim}

\begin{Verbatim}[commandchars=\\\{\}]
{\color{outcolor}Out[{\color{outcolor}3}]:}           user\_id  movie\_id  rating   timestamp  \textbackslash{}
        13937369    96288      2997     5.0  1159793149
        13937382    96288      4848     5.0  1159792275
        13937362    96288      2329     5.0  1159794208
        13937367    96288      2858     5.0  1159794335
        13937348    96288      1227     5.0  1159794784
        13937347    96288      1222     5.0  1159794563
        13937345    96288      1193     5.0  1159793167
        13937376    96288      3949     5.0  1159792156
        13937356    96288      1732     5.0  1159794733
        13937406    96288     26150     5.0  1159794905
        13937396    96288      5954     5.0  1159794332
        13937336    96288       296     5.0  1159792664
        13937334    96288       288     5.0  1159793780
        13937388    96288      4995     4.5  1159792982
        13937403    96288      7371     4.5  1159794656
        13937363    96288      2542     4.5  1159794369
        13937402    96288      7153     4.5  1159792638
        13937365    96288      2692     4.5  1159794799
        13937397    96288      5995     4.5  1159792380
        13937368    96288      2959     4.5  1159793322

                                                              title
        13937369                        Being John Malkovich (1999)
        13937382                            Mulholland Drive (2001)
        13937362                          American History X (1998)
        13937367                             American Beauty (1999)
        13937348                 Once Upon a Time in America (1984)
        13937347                           Full Metal Jacket (1987)
        13937345             One Flew Over the Cuckoo's Nest (1975)
        13937376                         Requiem for a Dream (2000)
        13937356                           Big Lebowski, The (1998)
        13937406              Andrei Rublev (Andrey Rublyov) (1969)
        13937396                                   25th Hour (2002)
        13937336                                Pulp Fiction (1994)
        13937334                        Natural Born Killers (1994)
        13937388                           Beautiful Mind, A (2001)
        13937403                                    Dogville (2003)
        13937363           Lock, Stock \& Two Smoking Barrels (1998)
        13937402  Lord of the Rings: The Return of the King, The{\ldots}
        13937365                   Run Lola Run (Lola rennt) (1998)
        13937397                                Pianist, The (2002)
        13937368                                  Fight Club (1999)
\end{Verbatim}

Programamos nuestro modelo de regresión y lo corremos para nuestros datos

    \begin{Verbatim}[commandchars=\\\{\}]
{\color{incolor}In [{\color{incolor}4}]:} \PY{k+kn}{from} \PY{n+nn}{scipy}\PY{n+nn}{.}\PY{n+nn}{optimize} \PY{k}{import} \PY{n}{fmin\PYZus{}cg}

        \PY{k}{def} \PY{n+nf}{fit\PYZus{}users}\PY{p}{(}\PY{n}{ratings}\PY{p}{,} \PY{n}{mask}\PY{p}{,} \PY{n}{features}\PY{p}{,} \PY{n}{means}\PY{o}{=}\PY{k+kc}{None}\PY{p}{,} \PY{n}{lam}\PY{o}{=}\PY{l+m+mi}{20}\PY{p}{)}\PY{p}{:}
            \PY{k}{if} \PY{n}{means} \PY{o+ow}{is} \PY{k+kc}{None}\PY{p}{:}
                \PY{n}{means} \PY{o}{=} \PY{n}{np}\PY{o}{.}\PY{n}{zeros}\PY{p}{(}\PY{n}{ratings}\PY{o}{.}\PY{n}{shape}\PY{p}{)}
            \PY{n}{r} \PY{o}{=} \PY{n}{mask}\PY{o}{.}\PY{n}{astype}\PY{p}{(}\PY{n+nb}{int}\PY{p}{)}\PY{o}{.}\PY{n}{T}
            \PY{n}{ratings} \PY{o}{=} \PY{n}{ratings}\PY{o}{.}\PY{n}{T}

            \PY{n}{theta\PYZus{}shape} \PY{o}{=} \PY{p}{(}\PY{n}{ratings}\PY{o}{.}\PY{n}{shape}\PY{p}{[}\PY{l+m+mi}{1}\PY{p}{]}\PY{p}{,} \PY{n}{features}\PY{o}{.}\PY{n}{shape}\PY{p}{[}\PY{l+m+mi}{1}\PY{p}{]}\PY{p}{)}

            \PY{n}{theta0} \PY{o}{=} \PY{n}{np}\PY{o}{.}\PY{n}{random}\PY{o}{.}\PY{n}{rand}\PY{p}{(}\PY{o}{*}\PY{n}{theta\PYZus{}shape}\PY{p}{)}\PY{o}{.}\PY{n}{flatten}\PY{p}{(}\PY{p}{)}

            \PY{k}{def} \PY{n+nf}{optimization\PYZus{}target}\PY{p}{(}\PY{n}{folded}\PY{p}{)}\PY{p}{:}
                \PY{n}{theta} \PY{o}{=} \PY{n}{np}\PY{o}{.}\PY{n}{reshape}\PY{p}{(}\PY{n}{folded}\PY{p}{,} \PY{n}{theta\PYZus{}shape}\PY{p}{)}
                \PY{n}{differences} \PY{o}{=} \PY{n}{r} \PY{o}{*} \PY{p}{(}\PY{n}{features} \PY{o}{@} \PY{n}{theta}\PY{o}{.}\PY{n}{T} \PY{o}{\PYZhy{}} \PY{n}{ratings}\PY{p}{)}
                \PY{k}{return} \PY{l+m+mf}{0.5} \PY{o}{*} \PY{n}{np}\PY{o}{.}\PY{n}{sum}\PY{p}{(}\PY{n}{differences}\PY{o}{*}\PY{o}{*}\PY{l+m+mi}{2}\PY{p}{)} \PY{o}{+} \PY{l+m+mf}{0.5} \PY{o}{*} \PY{n}{lam} \PY{o}{*} \PY{n}{np}\PY{o}{.}\PY{n}{sum}\PY{p}{(}\PY{n}{theta}\PY{o}{*}\PY{o}{*}\PY{l+m+mi}{2}\PY{p}{)}

            \PY{k}{def} \PY{n+nf}{gradient}\PY{p}{(}\PY{n}{folded}\PY{p}{)}\PY{p}{:}
                \PY{n}{theta} \PY{o}{=} \PY{n}{np}\PY{o}{.}\PY{n}{reshape}\PY{p}{(}\PY{n}{folded}\PY{p}{,} \PY{n}{theta\PYZus{}shape}\PY{p}{)}
                \PY{n}{differences} \PY{o}{=} \PY{n}{r} \PY{o}{*} \PY{p}{(}\PY{n}{features} \PY{o}{@} \PY{n}{theta}\PY{o}{.}\PY{n}{T} \PY{o}{\PYZhy{}} \PY{n}{ratings}\PY{p}{)}

                \PY{k}{return} \PY{p}{(}\PY{n}{differences}\PY{o}{.}\PY{n}{T} \PY{o}{@} \PY{n}{features} \PY{o}{+} \PY{n}{lam} \PY{o}{*} \PY{n}{theta}\PY{p}{)}\PY{o}{.}\PY{n}{flatten}\PY{p}{(}\PY{p}{)}

            \PY{n}{theta} \PY{o}{=} \PY{n}{fmin\PYZus{}cg}\PY{p}{(}\PY{n}{f}\PY{o}{=}\PY{n}{optimization\PYZus{}target}\PY{p}{,} \PY{n}{x0}\PY{o}{=}\PY{n}{theta0}\PY{p}{,} \PY{n}{fprime}\PY{o}{=}\PY{n}{gradient}\PY{p}{)}

            \PY{k}{return} \PY{n}{theta}

        \PY{n}{user\PYZus{}rating\PYZus{}vector} \PY{o}{=} \PY{n}{pd}\PY{o}{.}\PY{n}{Series}\PY{p}{(}\PY{n}{index}\PY{o}{=}\PY{n}{movies}\PY{o}{.}\PY{n}{index}\PY{p}{)}
        \PY{n}{user\PYZus{}rating\PYZus{}vector}\PY{p}{[}\PY{n}{user\PYZus{}ratings}\PY{o}{.}\PY{n}{movie\PYZus{}id}\PY{p}{]} \PY{o}{=} \PY{n}{user\PYZus{}ratings}\PY{o}{.}\PY{n}{rating}
        \PY{n}{user\PYZus{}ratings\PYZus{}matrix} \PY{o}{=} \PY{n}{user\PYZus{}rating\PYZus{}vector}\PY{o}{.}\PY{n}{values}\PY{o}{.}\PY{n}{reshape}\PY{p}{(}\PY{l+m+mi}{1}\PY{p}{,} \PY{o}{\PYZhy{}}\PY{l+m+mi}{1}\PY{p}{)}
        \PY{n}{ratings\PYZus{}mask} \PY{o}{=} \PY{n}{np}\PY{o}{.}\PY{n}{isfinite}\PY{p}{(}\PY{n}{user\PYZus{}ratings\PYZus{}matrix}\PY{p}{)}
        \PY{n}{user\PYZus{}ratings\PYZus{}matrix} \PY{o}{=} \PY{n}{np}\PY{o}{.}\PY{n}{nan\PYZus{}to\PYZus{}num}\PY{p}{(}\PY{n}{user\PYZus{}ratings\PYZus{}matrix}\PY{p}{)}

        \PY{n}{theta} \PY{o}{=} \PY{n}{fit\PYZus{}users}\PY{p}{(}\PY{n}{user\PYZus{}ratings\PYZus{}matrix}\PY{p}{,} \PY{n}{ratings\PYZus{}mask}\PY{p}{,} \PY{n}{genome\PYZus{}matrix}\PY{o}{.}\PY{n}{values}\PY{p}{)}
\end{Verbatim}


    \begin{Verbatim}[commandchars=\\\{\}]
Optimization terminated successfully.
         Current function value: 5.082708
         Iterations: 96
         Function evaluations: 210
         Gradient evaluations: 210

    \end{Verbatim}

Ahora podemos tomar el $\theta$ que obtuvimos como las preferencias de nuestro usuario

    \begin{Verbatim}[commandchars=\\\{\}]
{\color{incolor}In [{\color{incolor}5}]:} \PY{n}{preferences} \PY{o}{=} \PY{n}{pd}\PY{o}{.}\PY{n}{Series}\PY{p}{(}\PY{n}{theta}\PY{p}{,} \PY{n}{index}\PY{o}{=}\PY{n}{genome\PYZus{}matrix}\PY{o}{.}\PY{n}{columns}\PY{p}{)}

        \PY{n}{preferences}\PY{o}{.}\PY{n}{sort\PYZus{}values}\PY{p}{(}\PY{n}{ascending}\PY{o}{=}\PY{k+kc}{False}\PY{p}{)}\PY{o}{.}\PY{n}{head}\PY{p}{(}\PY{l+m+mi}{20}\PY{p}{)}
\end{Verbatim}


\begin{Verbatim}[commandchars=\\\{\}]
{\color{outcolor}Out[{\color{outcolor}5}]:} dark                  0.058778
        masterpiece           0.057257
        cult classic          0.056889
        innocence lost        0.049008
        oscar winner          0.048513
        mentor                0.048007
        mad scientist         0.043156
        imagination           0.041792
        imdb top 250          0.039944
        love                  0.039931
        dreams                0.039512
        gunfight              0.039445
        monster               0.038432
        childhood             0.037762
        prison                0.037117
        social commentary     0.036180
        weird                 0.035976
        reflective            0.035807
        great music           0.035077
        alternate universe    0.034496
        dtype: float64
\end{Verbatim}

Y podemos usar estas preferencias para darle recomendaciones

    \begin{Verbatim}[commandchars=\\\{\}]
{\color{incolor}In [{\color{incolor}6}]:} \PY{n}{predictions} \PY{o}{=} \PY{n}{pd}\PY{o}{.}\PY{n}{DataFrame}\PY{p}{(}\PY{n}{genome\PYZus{}matrix}\PY{o}{.}\PY{n}{values} \PY{o}{@} \PY{n}{theta}\PY{p}{,} \PY{n}{index}\PY{o}{=}\PY{n}{movies}\PY{o}{.}\PY{n}{index}\PY{p}{)}

        \PY{n}{predictions} \PY{o}{=} \PY{n}{predictions}\PY{o}{.}\PY{n}{assign}\PY{p}{(}\PY{n}{title}\PY{o}{=}\PY{n}{movies}\PY{o}{.}\PY{n}{title}\PY{p}{)}
        \PY{n}{predictions}\PY{o}{.}\PY{n}{columns} \PY{o}{=} \PY{p}{[}\PY{l+s+s1}{\PYZsq{}}\PY{l+s+s1}{rating}\PY{l+s+s1}{\PYZsq{}}\PY{p}{,} \PY{l+s+s1}{\PYZsq{}}\PY{l+s+s1}{title}\PY{l+s+s1}{\PYZsq{}}\PY{p}{]}
        \PY{n}{predictions}\PY{o}{.}\PY{n}{loc}\PY{p}{[}\PY{o}{\PYZti{}}\PY{n}{ratings\PYZus{}mask}\PY{o}{.}\PY{n}{flatten}\PY{p}{(}\PY{p}{)}\PY{p}{,} \PY{p}{:}\PY{p}{]} \PYZbs{}
            \PY{o}{.}\PY{n}{sort\PYZus{}values}\PY{p}{(}\PY{n}{by}\PY{o}{=}\PY{p}{[}\PY{l+s+s1}{\PYZsq{}}\PY{l+s+s1}{rating}\PY{l+s+s1}{\PYZsq{}}\PY{p}{]}\PY{p}{,} \PY{n}{ascending}\PY{o}{=}\PY{k+kc}{False}\PY{p}{)}\PY{o}{.}\PY{n}{head}\PY{p}{(}\PY{l+m+mi}{20}\PY{p}{)}
\end{Verbatim}


\begin{Verbatim}[commandchars=\\\{\}]
{\color{outcolor}Out[{\color{outcolor}6}]:}             rating                                              title
        movie\_id
        778       5.293866                               Trainspotting (1996)
        5147      5.287688         Wild Strawberries (Smultronstället) (1957)
        1237      5.286941    Seventh Seal, The (Sjunde inseglet, Det) (1957)
        111       5.256287                                 Taxi Driver (1976)
        1206      5.253421                         Clockwork Orange, A (1971)
        1251      5.228539                                  8 1/2 (8½) (1963)
        99764     5.175642                   It's Such a Beautiful Day (2012)
        7361      5.158738       Eternal Sunshine of the Spotless Mind (2004)
        7068      5.119508  Last Year at Marienbad (L'Année dernière à Mar{\ldots}
        1228      5.119437                                 Raging Bull (1980)
        1199      5.104201                                      Brazil (1985)
        25793     5.091687                                      Vampyr (1932)
        4878      5.089001                                Donnie Darko (2001)
        6440      5.058139                                 Barton Fink (1991)
        3476      5.040493                              Jacob's Ladder (1990)
        61240     5.029298  Let the Right One In (Låt den rätte komma in) {\ldots}
        4658      5.022652                                Santa Sangre (1989)
        3676      5.021929                                  Eraserhead (1977)
        1213      5.019151                                  Goodfellas (1990)
        55444     5.012329                                     Control (2007)
\end{Verbatim}

\subsection{F.C. con vecinos mas cercanos para \emph{MovieLens}}
\label{subsec:a3}

A continuación se presenta la implementación del metodo de filtrado colaborativo para vecinos mas cercanos
 para la base de datos de MovieLens. Primero cargamos los datos como lo hemos hecho antes:

\begin{Verbatim}[commandchars=\\\{\}]
{\color{incolor}In [{\color{incolor}1}]:} \PY{k+kn}{import} \PY{n+nn}{pandas} \PY{k}{as} \PY{n+nn}{pd}
        \PY{k+kn}{import} \PY{n+nn}{numpy} \PY{k}{as} \PY{n+nn}{np}

        \PY{n}{pd}\PY{o}{.}\PY{n}{read\PYZus{}csv}
        \PY{n}{ratings\PYZus{}df} \PY{o}{=} \PY{n}{pd}\PY{o}{.}\PY{n}{read\PYZus{}csv}\PY{p}{(}\PY{l+s+s1}{\PYZsq{}}\PY{l+s+s1}{datasets/ml\PYZhy{}latest\PYZhy{}small/ratings.csv}\PY{l+s+s1}{\PYZsq{}}\PY{p}{)}
        \PY{n}{movies} \PY{o}{=} \PY{n}{pd}\PY{o}{.}\PY{n}{read\PYZus{}csv}\PY{p}{(}\PY{l+s+s1}{\PYZsq{}}\PY{l+s+s1}{datasets/ml\PYZhy{}latest\PYZhy{}small/movies.csv}\PY{l+s+s1}{\PYZsq{}}\PY{p}{,} \PY{n}{index\PYZus{}col}\PY{o}{=}\PY{l+m+mi}{0}\PY{p}{)}

        \PY{n}{ratings\PYZus{}df}\PY{o}{.}\PY{n}{columns} \PY{o}{=} \PY{p}{[}\PY{l+s+s1}{\PYZsq{}}\PY{l+s+s1}{user\PYZus{}id}\PY{l+s+s1}{\PYZsq{}}\PY{p}{,} \PY{l+s+s1}{\PYZsq{}}\PY{l+s+s1}{movie\PYZus{}id}\PY{l+s+s1}{\PYZsq{}}\PY{p}{,} \PY{l+s+s1}{\PYZsq{}}\PY{l+s+s1}{rating}\PY{l+s+s1}{\PYZsq{}}\PY{p}{,} \PY{l+s+s1}{\PYZsq{}}\PY{l+s+s1}{timestamp}\PY{l+s+s1}{\PYZsq{}}\PY{p}{]}

        \PY{n}{user\PYZus{}ratings} \PY{o}{=} \PY{n}{ratings\PYZus{}df}\PY{o}{.}\PY{n}{pivot}\PY{p}{(}\PY{n}{index}\PY{o}{=}\PY{l+s+s1}{\PYZsq{}}\PY{l+s+s1}{movie\PYZus{}id}\PY{l+s+s1}{\PYZsq{}}\PY{p}{,} \PY{n}{columns}\PY{o}{=}\PY{l+s+s1}{\PYZsq{}}\PY{l+s+s1}{user\PYZus{}id}\PY{l+s+s1}{\PYZsq{}}\PY{p}{,} \PY{n}{values}\PY{o}{=}\PY{l+s+s1}{\PYZsq{}}\PY{l+s+s1}{rating}\PY{l+s+s1}{\PYZsq{}}\PY{p}{)}
\end{Verbatim}

Escogemos una columna al azar de la matriz, que representa las calificaciones del usuario correspondiente. Imprimimos sus peliculas mejor calificadas.

    \begin{Verbatim}[commandchars=\\\{\}]
{\color{incolor}In [{\color{incolor}2}]:} \PY{n}{user} \PY{o}{=} \PY{n}{user\PYZus{}ratings}\PY{o}{.}\PY{n}{sample}\PY{p}{(}\PY{n}{axis}\PY{o}{=}\PY{l+m+mi}{1}\PY{p}{)}\PY{o}{.}\PY{n}{assign}\PY{p}{(}\PY{n}{title}\PY{o}{=}\PY{n}{movies}\PY{o}{.}\PY{n}{title}\PY{p}{[}\PY{n}{user\PYZus{}ratings}\PY{o}{.}\PY{n}{index}\PY{p}{]}\PY{p}{)}
         \PY{n}{user}\PY{o}{.}\PY{n}{columns} \PY{o}{=} \PY{p}{[}\PY{l+s+s1}{\PYZsq{}}\PY{l+s+s1}{rating}\PY{l+s+s1}{\PYZsq{}}\PY{p}{,} \PY{l+s+s1}{\PYZsq{}}\PY{l+s+s1}{title}\PY{l+s+s1}{\PYZsq{}}\PY{p}{]}
         \PY{n}{user}\PY{o}{.}\PY{n}{sort\PYZus{}values}\PY{p}{(}\PY{n}{by}\PY{o}{=}\PY{l+s+s1}{\PYZsq{}}\PY{l+s+s1}{rating}\PY{l+s+s1}{\PYZsq{}}\PY{p}{,} \PY{n}{ascending}\PY{o}{=}\PY{k+kc}{False}\PY{p}{)}\PY{o}{.}\PY{n}{head}\PY{p}{(}\PY{l+m+mi}{20}\PY{p}{)}
\end{Verbatim}


\begin{Verbatim}[commandchars=\\\{\}]
{\color{outcolor}Out[{\color{outcolor}2}]:}           rating                                              title
         movie\_id
         1258         5.0                                Shining, The (1980)
         1255         5.0                                   Bad Taste (1987)
         1349         5.0  Vampire in Venice (Nosferatu a Venezia) (Nosfe{\ldots}
         1348         5.0  Nosferatu (Nosferatu, eine Symphonie des Graue{\ldots}
         1339         5.0             Dracula (Bram Stoker's Dracula) (1992)
         1387         5.0                                        Jaws (1975)
         247          5.0                          Heavenly Creatures (1994)
         1333         4.0                                  Birds, The (1963)
         1261         4.0                 Evil Dead II (Dead by Dawn) (1987)
         1343         4.0                                   Cape Fear (1991)
         1340         4.0  Bride of Frankenstein, The (Bride of Frankenst{\ldots}
         188          4.0                               Prophecy, The (1995)
         1321         4.0             American Werewolf in London, An (1981)
         1            4.0                                   Toy Story (1995)
         1219         4.0                                      Psycho (1960)
         1215         4.0                            Army of Darkness (1993)
         25           3.0                           Leaving Las Vegas (1995)
         296          3.0                                Pulp Fiction (1994)
         1341         3.0                             Burnt Offerings (1976)
         593          3.0                   Silence of the Lambs, The (1991)
\end{Verbatim}

Ahora, inicializamos nuestro algoritmo de aprendizaje.

    \begin{Verbatim}[commandchars=\\\{\}]
{\color{incolor}In [{\color{incolor}3}]:} \PY{k+kn}{from} \PY{n+nn}{sklearn}\PY{n+nn}{.}\PY{n+nn}{neighbors} \PY{k}{import} \PY{n}{NearestNeighbors}
         \PY{n}{ratings\PYZus{}mask} \PY{o}{=} \PY{n}{np}\PY{o}{.}\PY{n}{isfinite}\PY{p}{(}\PY{n}{user\PYZus{}ratings}\PY{o}{.}\PY{n}{values}\PY{p}{)}\PY{o}{.}\PY{n}{T}
         \PY{n}{user\PYZus{}matrix} \PY{o}{=} \PY{n}{np}\PY{o}{.}\PY{n}{nan\PYZus{}to\PYZus{}num}\PY{p}{(}\PY{n}{user\PYZus{}ratings}\PY{o}{.}\PY{n}{values}\PY{p}{)}\PY{o}{.}\PY{n}{T}

         \PY{n}{nnbrs} \PY{o}{=} \PY{n}{NearestNeighbors}\PY{p}{(}\PY{n}{n\PYZus{}neighbors}\PY{o}{=}\PY{l+m+mi}{5}\PY{p}{,} \PY{n}{algorithm}\PY{o}{=}\PY{l+s+s1}{\PYZsq{}}\PY{l+s+s1}{brute}\PY{l+s+s1}{\PYZsq{}}\PY{p}{,} \PY{n}{metric}\PY{o}{=}\PY{l+s+s1}{\PYZsq{}}\PY{l+s+s1}{cosine}\PY{l+s+s1}{\PYZsq{}}\PY{p}{)}\PY{o}{.}\PY{n}{fit}\PY{p}{(}\PY{n}{user\PYZus{}matrix}\PY{p}{)}
\end{Verbatim}

Obtenemos los vecinos de nuestro usuario y aproximamos sus calificaciones tomando en cuenta las calificaciones de sus vecinos.

    \begin{Verbatim}[commandchars=\\\{\}]
{\color{incolor}In [{\color{incolor}4}]:} \PY{n}{distance}\PY{p}{,} \PY{n}{neighbors} \PY{o}{=} \PY{n}{nnbrs}\PY{o}{.}\PY{n}{kneighbors}\PY{p}{(}\PY{n}{np}\PY{o}{.}\PY{n}{nan\PYZus{}to\PYZus{}num}\PY{p}{(}\PY{n}{user}\PY{o}{.}\PY{n}{rating}\PY{o}{.}\PY{n}{values}\PY{o}{.}\PY{n}{reshape}\PY{p}{(}\PY{l+m+mi}{1}\PY{p}{,} \PY{o}{\PYZhy{}}\PY{l+m+mi}{1}\PY{p}{)}\PY{p}{)}\PY{p}{)}

         \PY{n}{similarity} \PY{o}{=} \PY{l+m+mi}{1} \PY{o}{\PYZhy{}} \PY{n}{distance}\PY{o}{.}\PY{n}{flatten}\PY{p}{(}\PY{p}{)}\PY{p}{[}\PY{l+m+mi}{1}\PY{p}{:}\PY{p}{]}

         \PY{n}{neighbor\PYZus{}mask} \PY{o}{=} \PY{n}{ratings\PYZus{}mask}\PY{p}{[}\PY{n}{neighbors}\PY{o}{.}\PY{n}{flatten}\PY{p}{(}\PY{p}{)}\PY{p}{[}\PY{l+m+mi}{1}\PY{p}{:}\PY{p}{]}\PY{p}{]}
         \PY{n}{neighbors} \PY{o}{=} \PY{n}{user\PYZus{}matrix}\PY{p}{[}\PY{n}{neighbors}\PY{o}{.}\PY{n}{flatten}\PY{p}{(}\PY{p}{)}\PY{p}{[}\PY{l+m+mi}{1}\PY{p}{:}\PY{p}{]}\PY{p}{]}

         \PY{n}{np}\PY{o}{.}\PY{n}{seterr}\PY{p}{(}\PY{n}{divide}\PY{o}{=}\PY{l+s+s1}{\PYZsq{}}\PY{l+s+s1}{ignore}\PY{l+s+s1}{\PYZsq{}}\PY{p}{,} \PY{n}{invalid}\PY{o}{=}\PY{l+s+s1}{\PYZsq{}}\PY{l+s+s1}{ignore}\PY{l+s+s1}{\PYZsq{}}\PY{p}{)}

         \PY{n}{pred} \PY{o}{=} \PY{n}{similarity} \PY{o}{@} \PY{n}{neighbors}
         \PY{n}{pred} \PY{o}{=} \PY{n}{pred} \PY{o}{/} \PY{p}{(}\PY{n}{similarity} \PY{o}{*} \PY{n}{neighbor\PYZus{}mask}\PY{o}{.}\PY{n}{T}\PY{p}{)}\PY{o}{.}\PY{n}{sum}\PY{p}{(}\PY{n}{axis}\PY{o}{=}\PY{l+m+mi}{1}\PY{p}{)}

         \PY{n}{pred} \PY{o}{=} \PY{n}{np}\PY{o}{.}\PY{n}{nan\PYZus{}to\PYZus{}num}\PY{p}{(}\PY{n}{pred}\PY{p}{)}\PY{o}{.}\PY{n}{flatten}\PY{p}{(}\PY{p}{)}
\end{Verbatim}


Finalmente, para tener una idea de nuestros resultados, imprimimos las películas con mayor calificación predicha.

    \begin{Verbatim}[commandchars=\\\{\}]
{\color{incolor}In [{\color{incolor}5}]:} \PY{n}{user}\PY{p}{[}\PY{l+s+s1}{\PYZsq{}}\PY{l+s+s1}{predicted}\PY{l+s+s1}{\PYZsq{}}\PY{p}{]} \PY{o}{=} \PY{n}{pred}
         \PY{n}{user}\PY{o}{.}\PY{n}{sort\PYZus{}values}\PY{p}{(}\PY{n}{by}\PY{o}{=}\PY{l+s+s1}{\PYZsq{}}\PY{l+s+s1}{predicted}\PY{l+s+s1}{\PYZsq{}}\PY{p}{,} \PY{n}{ascending}\PY{o}{=}\PY{k+kc}{False}\PY{p}{)}\PY{o}{.}\PY{n}{head}\PY{p}{(}\PY{l+m+mi}{20}\PY{p}{)}
\end{Verbatim}


\begin{Verbatim}[commandchars=\\\{\}]
{\color{outcolor}Out[{\color{outcolor}5}]:}           rating                                              title  predicted
         movie\_id
         2650         NaN                  Ghost of Frankenstein, The (1942)        5.0
         2652         NaN                  Curse of Frankenstein, The (1957)        5.0
         2634         NaN                                  Mummy, The (1959)        5.0
         2644         NaN                                     Dracula (1931)        5.0
         1721         NaN                                     Titanic (1997)        5.0
         2647         NaN                       House of Frankenstein (1944)        5.0
         2648         NaN                                Frankenstein (1931)        5.0
         2649         NaN                         Son of Frankenstein (1939)        5.0
         2636         NaN                          Mummy's Ghost, The (1944)        5.0
         2651         NaN             Frankenstein Meets the Wolf Man (1943)        5.0
         2635         NaN                          Mummy's Curse, The (1944)        5.0
         2654         NaN                               Wolf Man, The (1941)        5.0
         2664         NaN              Invasion of the Body Snatchers (1956)        5.0
         2716         NaN         Ghostbusters (a.k.a. Ghost Busters) (1984)        5.0
         3018         NaN                                 Re-Animator (1985)        5.0
         1348         5.0  Nosferatu (Nosferatu, eine Symphonie des Graue{\ldots}        5.0
         1354         NaN                          Breaking the Waves (1996)        5.0
         1219         4.0                                      Psycho (1960)        5.0
         326          NaN                            To Live (Huozhe) (1994)        5.0
         534          NaN                                 Shadowlands (1993)        5.0
\end{Verbatim}

\subsection{Recomendaciones de MovieLens usando filtrado colaborativo con regresion}
\label{subsec:a4}

Comenzamos implementando el algoritmo de regresión para todos nuestro parametros.

\begin{Verbatim}[commandchars=\\\{\}]
{\color{incolor}In [{\color{incolor}1}]:} \PY{k+kn}{import} \PY{n+nn}{pandas} \PY{k}{as} \PY{n+nn}{pd}
        \PY{k+kn}{import} \PY{n+nn}{numpy} \PY{k}{as} \PY{n+nn}{np}
        \PY{k+kn}{from} \PY{n+nn}{scipy}\PY{n+nn}{.}\PY{n+nn}{optimize} \PY{k}{import} \PY{n}{fmin\PYZus{}cg}


        \PY{k}{def} \PY{n+nf}{simple\PYZus{}fit}\PY{p}{(}\PY{n}{ui\PYZus{}matrix}\PY{p}{,} \PY{n}{r\PYZus{}matrix}\PY{p}{,} \PY{n}{num\PYZus{}features}\PY{p}{,} \PY{n}{lam}\PY{p}{)}\PY{p}{:}
            \PY{l+s+sd}{\PYZdq{}\PYZdq{}\PYZdq{}}
        \PY{l+s+sd}{    Ajusta el modelo de regresion dada una matriz de calificaciones y}
        \PY{l+s+sd}{    una mascara de calificados}
        \PY{l+s+sd}{    \PYZdq{}\PYZdq{}\PYZdq{}}
            \PY{n}{y} \PY{o}{=} \PY{n}{ui\PYZus{}matrix}
            \PY{n}{r} \PY{o}{=} \PY{n}{r\PYZus{}matrix}
            \PY{n}{num\PYZus{}items}\PY{p}{,} \PY{n}{num\PYZus{}users} \PY{o}{=} \PY{n}{y}\PY{o}{.}\PY{n}{shape}

            \PY{n}{theta0} \PY{o}{=} \PY{n}{np}\PY{o}{.}\PY{n}{random}\PY{o}{.}\PY{n}{rand}\PY{p}{(}\PY{n}{num\PYZus{}users}\PY{p}{,} \PY{n}{num\PYZus{}features}\PY{p}{)}
            \PY{n}{x0} \PY{o}{=} \PY{n}{np}\PY{o}{.}\PY{n}{random}\PY{o}{.}\PY{n}{rand}\PY{p}{(}\PY{n}{num\PYZus{}items}\PY{p}{,} \PY{n}{num\PYZus{}features}\PY{p}{)}

            \PY{k}{def} \PY{n+nf}{fold\PYZus{}matrices}\PY{p}{(}\PY{n}{x\PYZus{}matrix}\PY{p}{,} \PY{n}{theta\PYZus{}matrix}\PY{p}{)}\PY{p}{:}
                \PY{k}{return} \PY{n}{np}\PY{o}{.}\PY{n}{concatenate}\PY{p}{(}\PY{p}{[}\PY{n}{x\PYZus{}matrix}\PY{o}{.}\PY{n}{flatten}\PY{p}{(}\PY{p}{)}\PY{p}{,} \PY{n}{theta\PYZus{}matrix}\PY{o}{.}\PY{n}{flatten}\PY{p}{(}\PY{p}{)}\PY{p}{]}\PY{p}{)}

            \PY{k}{def} \PY{n+nf}{unfold\PYZus{}vector}\PY{p}{(}\PY{n}{x}\PY{p}{)}\PY{p}{:}
                \PY{n}{x\PYZus{}matrix} \PY{o}{=} \PY{n}{np}\PY{o}{.}\PY{n}{reshape}\PY{p}{(}\PY{n}{x}\PY{p}{[}\PY{p}{:}\PY{n}{x0}\PY{o}{.}\PY{n}{size}\PY{p}{]}\PY{p}{,}
                                      \PY{n}{x0}\PY{o}{.}\PY{n}{shape}\PY{p}{)}
                \PY{n}{theta\PYZus{}matrix} \PY{o}{=} \PY{n}{np}\PY{o}{.}\PY{n}{reshape}\PY{p}{(}\PY{n}{x}\PY{p}{[}\PY{n}{x0}\PY{o}{.}\PY{n}{size}\PY{p}{:}\PY{p}{]}\PY{p}{,}
                                          \PY{n}{theta0}\PY{o}{.}\PY{n}{shape}\PY{p}{)}
                \PY{k}{return} \PY{n}{x\PYZus{}matrix}\PY{p}{,} \PY{n}{theta\PYZus{}matrix}

            \PY{k}{def} \PY{n+nf}{unfold\PYZus{}parameter}\PY{p}{(}\PY{n}{f}\PY{p}{)}\PY{p}{:}
                \PY{k}{def} \PY{n+nf}{wrapper}\PY{p}{(}\PY{n}{x}\PY{p}{)}\PY{p}{:}
                    \PY{k}{return} \PY{n}{f}\PY{p}{(}\PY{o}{*}\PY{n}{unfold\PYZus{}vector}\PY{p}{(}\PY{n}{x}\PY{p}{)}\PY{p}{)}

                \PY{k}{return} \PY{n}{wrapper}

            \PY{n+nd}{@unfold\PYZus{}parameter}
            \PY{k}{def} \PY{n+nf}{optimization\PYZus{}target}\PY{p}{(}\PY{n}{x}\PY{p}{,} \PY{n}{theta}\PY{p}{)}\PY{p}{:}
                \PY{n}{differences} \PY{o}{=} \PY{n}{r} \PY{o}{*} \PY{p}{(}\PY{n}{x} \PY{o}{@} \PY{n}{theta}\PY{o}{.}\PY{n}{T} \PY{o}{\PYZhy{}} \PY{n}{y}\PY{p}{)}
                \PY{n}{square\PYZus{}error} \PY{o}{=} \PY{p}{(}\PY{l+m+mf}{0.5}\PY{p}{)} \PY{o}{*} \PY{n}{np}\PY{o}{.}\PY{n}{sum}\PY{p}{(}\PY{n}{differences}\PY{o}{*}\PY{o}{*}\PY{l+m+mi}{2}\PY{p}{)}
                \PY{n}{regularization} \PY{o}{=} \PY{p}{(}\PY{n}{lam} \PY{o}{/} \PY{l+m+mi}{2}\PY{p}{)} \PY{o}{*} \PY{p}{(}\PY{n}{np}\PY{o}{.}\PY{n}{sum}\PY{p}{(}\PY{n}{x}\PY{o}{*}\PY{o}{*}\PY{l+m+mi}{2}\PY{p}{)} \PY{o}{+} \PY{n}{np}\PY{o}{.}\PY{n}{sum}\PY{p}{(}\PY{n}{x}\PY{o}{*}\PY{o}{*}\PY{l+m+mi}{2}\PY{p}{)}\PY{p}{)}

                \PY{k}{return} \PY{n}{square\PYZus{}error} \PY{o}{+} \PY{n}{regularization}

            \PY{n+nd}{@unfold\PYZus{}parameter}
            \PY{k}{def} \PY{n+nf}{gradient}\PY{p}{(}\PY{n}{x}\PY{p}{,} \PY{n}{theta}\PY{p}{)}\PY{p}{:}
                \PY{n}{differences} \PY{o}{=} \PY{n}{np}\PY{o}{.}\PY{n}{multiply}\PY{p}{(}\PY{p}{(}\PY{n}{np}\PY{o}{.}\PY{n}{dot}\PY{p}{(}\PY{n}{x}\PY{p}{,} \PY{n}{theta}\PY{o}{.}\PY{n}{T}\PY{p}{)} \PY{o}{\PYZhy{}} \PY{n}{y}\PY{p}{)}\PY{p}{,} \PY{n}{r}\PY{p}{)}
                \PY{n}{x\PYZus{}grad} \PY{o}{=} \PY{n}{np}\PY{o}{.}\PY{n}{dot}\PY{p}{(}\PY{n}{differences}\PY{p}{,} \PY{n}{theta}\PY{p}{)} \PY{o}{+} \PY{n}{lam} \PY{o}{*} \PY{n}{x}
                \PY{n}{theta\PYZus{}grad} \PY{o}{=} \PY{n}{np}\PY{o}{.}\PY{n}{dot}\PY{p}{(}\PY{n}{x}\PY{o}{.}\PY{n}{T}\PY{p}{,} \PY{n}{differences}\PY{p}{)}\PY{o}{.}\PY{n}{T} \PY{o}{+} \PY{n}{lam} \PY{o}{*} \PY{n}{theta}

                \PY{k}{return} \PY{n}{fold\PYZus{}matrices}\PY{p}{(}\PY{n}{x\PYZus{}grad}\PY{p}{,} \PY{n}{theta\PYZus{}grad}\PY{p}{)}

            \PY{n}{init\PYZus{}fold} \PY{o}{=} \PY{n}{fold\PYZus{}matrices}\PY{p}{(}\PY{n}{x0}\PY{p}{,} \PY{n}{theta0}\PY{p}{)}
            \PY{n}{result} \PY{o}{=} \PY{n}{fmin\PYZus{}cg}\PY{p}{(}\PY{n}{f}\PY{o}{=}\PY{n}{optimization\PYZus{}target}\PY{p}{,} \PY{n}{x0}\PY{o}{=}\PY{n}{init\PYZus{}fold}\PY{p}{,} \PY{n}{fprime}\PY{o}{=}\PY{n}{gradient}\PY{p}{)}

            \PY{n}{x}\PY{p}{,} \PY{n}{theta} \PY{o}{=} \PY{n}{unfold\PYZus{}vector}\PY{p}{(}\PY{n}{result}\PY{p}{)}

            \PY{k}{return} \PY{n}{x}\PY{p}{,} \PY{n}{theta}
\end{Verbatim}


Ahora, envolvemos el algoritmo en una funcion de normalización para obtener los resultados incluyendo normalización de medias.

    \begin{Verbatim}[commandchars=\\\{\}]
{\color{incolor}In [{\color{incolor}2}]:} \PY{k}{def} \PY{n+nf}{normalized\PYZus{}fit}\PY{p}{(}\PY{n}{y}\PY{p}{,} \PY{o}{*}\PY{n}{args}\PY{p}{)}\PY{p}{:}
            \PY{n}{means} \PY{o}{=} \PY{n}{np}\PY{o}{.}\PY{n}{nanmean}\PY{p}{(}\PY{n}{y}\PY{p}{,} \PY{n}{axis}\PY{o}{=}\PY{l+m+mi}{1}\PY{p}{)}
            \PY{n}{y} \PY{o}{=} \PY{n}{y} \PY{o}{\PYZhy{}} \PY{n}{means}\PY{o}{.}\PY{n}{reshape}\PY{p}{(}\PY{o}{\PYZhy{}}\PY{l+m+mi}{1}\PY{p}{,} \PY{l+m+mi}{1}\PY{p}{)}

            \PY{n}{r} \PY{o}{=} \PY{o}{\PYZhy{}}\PY{p}{(}\PY{n}{np}\PY{o}{.}\PY{n}{isnan}\PY{p}{(}\PY{n}{y}\PY{p}{)}\PY{o}{.}\PY{n}{astype}\PY{p}{(}\PY{n+nb}{int}\PY{p}{)} \PY{o}{\PYZhy{}} \PY{l+m+mi}{1}\PY{p}{)}
            \PY{n}{y} \PY{o}{=} \PY{n}{np}\PY{o}{.}\PY{n}{nan\PYZus{}to\PYZus{}num}\PY{p}{(}\PY{n}{y}\PY{p}{)}

            \PY{n}{x}\PY{p}{,} \PY{n}{theta} \PY{o}{=} \PY{n}{simple\PYZus{}fit}\PY{p}{(}\PY{n}{y}\PY{p}{,} \PY{n}{r}\PY{p}{,} \PY{o}{*}\PY{n}{args}\PY{p}{)}

            \PY{k}{return} \PY{n}{x}\PY{p}{,} \PY{n}{theta}\PY{p}{,} \PY{n}{means}
\end{Verbatim}


Como de costumbre, cargamos el conjunto de datos de MovieLens

    \begin{Verbatim}[commandchars=\\\{\}]
{\color{incolor}In [{\color{incolor}3}]:} \PY{n}{ratings\PYZus{}df} \PY{o}{=} \PY{n}{pd}\PY{o}{.}\PY{n}{read\PYZus{}csv}\PY{p}{(}\PY{l+s+s1}{\PYZsq{}}\PY{l+s+s1}{datasets/ml\PYZhy{}latest\PYZhy{}small/ratings.csv}\PY{l+s+s1}{\PYZsq{}}\PY{p}{)}
        \PY{n}{movies} \PY{o}{=} \PY{n}{pd}\PY{o}{.}\PY{n}{read\PYZus{}csv}\PY{p}{(}\PY{l+s+s1}{\PYZsq{}}\PY{l+s+s1}{datasets/ml\PYZhy{}latest\PYZhy{}small/movies.csv}\PY{l+s+s1}{\PYZsq{}}\PY{p}{,} \PY{n}{index\PYZus{}col}\PY{o}{=}\PY{l+m+mi}{0}\PY{p}{)}

        \PY{n}{ratings\PYZus{}df}\PY{o}{.}\PY{n}{columns} \PY{o}{=} \PY{p}{[}\PY{l+s+s1}{\PYZsq{}}\PY{l+s+s1}{user\PYZus{}id}\PY{l+s+s1}{\PYZsq{}}\PY{p}{,} \PY{l+s+s1}{\PYZsq{}}\PY{l+s+s1}{movie\PYZus{}id}\PY{l+s+s1}{\PYZsq{}}\PY{p}{,} \PY{l+s+s1}{\PYZsq{}}\PY{l+s+s1}{rating}\PY{l+s+s1}{\PYZsq{}}\PY{p}{,} \PY{l+s+s1}{\PYZsq{}}\PY{l+s+s1}{timestamp}\PY{l+s+s1}{\PYZsq{}}\PY{p}{]}

        \PY{n}{user\PYZus{}ratings} \PY{o}{=} \PY{n}{ratings\PYZus{}df}\PY{o}{.}\PY{n}{pivot}\PY{p}{(}\PY{n}{index}\PY{o}{=}\PY{l+s+s1}{\PYZsq{}}\PY{l+s+s1}{movie\PYZus{}id}\PY{l+s+s1}{\PYZsq{}}\PY{p}{,} \PY{n}{columns}\PY{o}{=}\PY{l+s+s1}{\PYZsq{}}\PY{l+s+s1}{user\PYZus{}id}\PY{l+s+s1}{\PYZsq{}}\PY{p}{,} \PY{n}{values}\PY{o}{=}\PY{l+s+s1}{\PYZsq{}}\PY{l+s+s1}{rating}\PY{l+s+s1}{\PYZsq{}}\PY{p}{)}
\end{Verbatim}

Ajustamos el modelo de regresión para la matriz usando 200 caracteristicas y un valor de regularización de 0.2.

    \begin{Verbatim}[commandchars=\\\{\}]
{\color{incolor}In [{\color{incolor}4}]:} \PY{n}{x}\PY{p}{,} \PY{n}{theta}\PY{p}{,} \PY{n}{means} \PY{o}{=} \PY{n}{normalized\PYZus{}fit}\PY{p}{(}\PY{n}{user\PYZus{}ratings}\PY{o}{.}\PY{n}{values}\PY{p}{,} \PY{l+m+mi}{200}\PY{p}{,} \PY{l+m+mf}{0.2}\PY{p}{)}

        \PY{n}{feature\PYZus{}df} \PY{o}{=} \PY{n}{pd}\PY{o}{.}\PY{n}{DataFrame}\PY{p}{(}\PY{n}{x}\PY{p}{,} \PY{n}{index}\PY{o}{=}\PY{n}{user\PYZus{}ratings}\PY{o}{.}\PY{n}{index}\PY{p}{)}
\end{Verbatim}


    \begin{Verbatim}[commandchars=\\\{\}]
Warning: Desired error not necessarily achieved due to precision loss.
         Current function value: 915.220243
         Iterations: 1381
         Function evaluations: 2409
         Gradient evaluations: 2397

    \end{Verbatim}

Seleccionamos un usuario al azar y vemos sus calificaciones mas altas.

    \begin{Verbatim}[commandchars=\\\{\}]
{\color{incolor}In [{\color{incolor}30}]:} \PY{n}{user} \PY{o}{=} \PY{n}{user\PYZus{}ratings}\PY{o}{.}\PY{n}{sample}\PY{p}{(}\PY{n}{axis}\PY{o}{=}\PY{l+m+mi}{1}\PY{p}{)}
         \PY{n}{user\PYZus{}id} \PY{o}{=} \PY{n}{user}\PY{o}{.}\PY{n}{columns}\PY{p}{[}\PY{l+m+mi}{0}\PY{p}{]}
         \PY{n}{user} \PY{o}{=} \PY{n}{user}\PY{o}{.}\PY{n}{assign}\PY{p}{(}\PY{n}{title}\PY{o}{=}\PY{n}{movies}\PY{o}{.}\PY{n}{title}\PY{p}{[}\PY{n}{user\PYZus{}ratings}\PY{o}{.}\PY{n}{index}\PY{p}{]}\PY{p}{)}
         \PY{n}{user}\PY{o}{.}\PY{n}{columns} \PY{o}{=} \PY{p}{[}\PY{l+s+s1}{\PYZsq{}}\PY{l+s+s1}{rating}\PY{l+s+s1}{\PYZsq{}}\PY{p}{,} \PY{l+s+s1}{\PYZsq{}}\PY{l+s+s1}{title}\PY{l+s+s1}{\PYZsq{}}\PY{p}{]}
         \PY{n}{user}\PY{o}{.}\PY{n}{sort\PYZus{}values}\PY{p}{(}\PY{n}{by}\PY{o}{=}\PY{l+s+s1}{\PYZsq{}}\PY{l+s+s1}{rating}\PY{l+s+s1}{\PYZsq{}}\PY{p}{,} \PY{n}{ascending}\PY{o}{=}\PY{k+kc}{False}\PY{p}{)}\PY{o}{.}\PY{n}{head}\PY{p}{(}\PY{l+m+mi}{20}\PY{p}{)}
\end{Verbatim}


\begin{Verbatim}[commandchars=\\\{\}]
{\color{outcolor}Out[{\color{outcolor}30}]:}           rating                                              title
         movie\_id
         47           5.0                        Seven (a.k.a. Se7en) (1995)
         318          5.0                   Shawshank Redemption, The (1994)
         1704         5.0                           Good Will Hunting (1997)
         1196         5.0  Star Wars: Episode V - The Empire Strikes Back{\ldots}
         1387         5.0                                        Jaws (1975)
         1407         5.0                                      Scream (1996)
         1625         5.0                                   Game, The (1997)
         1617         5.0                           L.A. Confidential (1997)
         1689         4.0                Man Who Knew Too Little, The (1997)
         1672         4.0                              Rainmaker, The (1997)
         1619         4.0                        Seven Years in Tibet (1997)
         1597         4.0                           Conspiracy Theory (1997)
         1584         4.0                                     Contact (1997)
         1438         4.0                                Dante's Peak (1997)
         953          4.0                       It's a Wonderful Life (1946)
         1183         4.0                        English Patient, The (1996)
         1198         4.0  Raiders of the Lost Ark (Indiana Jones and the{\ldots}
         1488         3.0                            Devil's Own, The (1997)
         1687         3.0                                 Jackal, The (1997)
         1686         3.0                                  Red Corner (1997)
\end{Verbatim}

Calculamos las calificaciones aproximadas del usuario como $X^T\theta^{(j)} + \bar{X}$

    \begin{Verbatim}[commandchars=\\\{\}]
{\color{incolor}In [{\color{incolor}31}]:} \PY{n}{theta\PYZus{}df} \PY{o}{=} \PY{n}{pd}\PY{o}{.}\PY{n}{DataFrame}\PY{p}{(}\PY{n}{theta}\PY{p}{,} \PY{n}{index}\PY{o}{=}\PY{n}{user\PYZus{}ratings}\PY{o}{.}\PY{n}{columns}\PY{p}{)}
         \PY{n}{user\PYZus{}theta} \PY{o}{=} \PY{n}{theta\PYZus{}df}\PY{o}{.}\PY{n}{loc}\PY{p}{[}\PY{n}{user\PYZus{}id}\PY{p}{]}

         \PY{n}{pred} \PY{o}{=} \PY{p}{(}\PY{n}{user\PYZus{}theta}\PY{o}{.}\PY{n}{values} \PY{o}{@} \PY{n}{x}\PY{o}{.}\PY{n}{T}\PY{p}{)} \PY{o}{+} \PY{n}{means}
\end{Verbatim}


    \begin{Verbatim}[commandchars=\\\{\}]
{\color{incolor}In [{\color{incolor}32}]:} \PY{n}{user}\PY{p}{[}\PY{l+s+s1}{\PYZsq{}}\PY{l+s+s1}{predicted}\PY{l+s+s1}{\PYZsq{}}\PY{p}{]} \PY{o}{=} \PY{n}{pred}
         \PY{n}{user}\PY{o}{.}\PY{n}{sort\PYZus{}values}\PY{p}{(}\PY{n}{by}\PY{o}{=}\PY{l+s+s1}{\PYZsq{}}\PY{l+s+s1}{predicted}\PY{l+s+s1}{\PYZsq{}}\PY{p}{,} \PY{n}{ascending}\PY{o}{=}\PY{k+kc}{False}\PY{p}{)}\PY{o}{.}\PY{n}{head}\PY{p}{(}\PY{l+m+mi}{20}\PY{p}{)}
\end{Verbatim}


\begin{Verbatim}[commandchars=\\\{\}]
{\color{outcolor}Out[{\color{outcolor}32}]:}           rating                                              title  predicted
         movie\_id
         593          NaN                   Silence of the Lambs, The (1991)   5.344535
         2571         NaN                                 Matrix, The (1999)   5.034324
         4088         NaN                               Big Town, The (1987)   5.000031
         92494        NaN                        Dylan Moran: Monster (2004)   5.000028
         5960         NaN                               Bad Influence (1990)   5.000025
         3216         NaN             Vampyros Lesbos (Vampiras, Las) (1971)   5.000024
         4617         NaN                                 Let It Ride (1989)   5.000023
         4522         NaN                                  Masquerade (1988)   5.000022
         3038         NaN                        Face in the Crowd, A (1957)   5.000014
         26151        NaN                         Au Hasard Balthazar (1966)   5.000005
         107412       NaN  Kidnapping, Caucasian Style (Kavkazskaya plenn{\ldots}   5.000005
         62115        NaN                                 Six Shooter (2004)   5.000004
         118468       NaN                       Mei and the Kittenbus (2002)   5.000004
         961          NaN                      Little Lord Fauntleroy (1936)   5.000004
         1531         NaN                                Losing Chase (1996)   5.000003
         50703        NaN                                 Secret, The (2006)   5.000003
         3281         NaN                    Brandon Teena Story, The (1998)   5.000003
         32460        NaN                   Knockin' on Heaven's Door (1997)   5.000003
         107559       NaN  Am Ende eiens viel zu kurzen Tages (Death of a{\ldots}   5.000003
         3612         NaN  The Slipper and the Rose: The Story of Cindere{\ldots}   5.000003
\end{Verbatim}

\subsection{Filtrado colaborativo con descomposicion SVD}
\label{subsec:a5}
Usemos SVD para acelerar nuestro filtrado colaborativo de películas. Cargamos los datos y preparamos nuestra matriz con normalización de medias.

    \begin{Verbatim}[commandchars=\\\{\}]
{\color{incolor}In [{\color{incolor}1}]:} \PY{k+kn}{import} \PY{n+nn}{pandas} \PY{k}{as} \PY{n+nn}{pd}
        \PY{k+kn}{import} \PY{n+nn}{numpy} \PY{k}{as} \PY{n+nn}{np}

        \PY{n}{movies} \PY{o}{=} \PY{n}{pd}\PY{o}{.}\PY{n}{read\PYZus{}csv}\PY{p}{(}\PY{l+s+s1}{\PYZsq{}}\PY{l+s+s1}{datasets/ml\PYZhy{}latest\PYZhy{}small/movies.csv}\PY{l+s+s1}{\PYZsq{}}\PY{p}{,} \PY{n}{index\PYZus{}col}\PY{o}{=}\PY{l+m+mi}{0}\PY{p}{)}
        \PY{n}{ratings\PYZus{}df} \PY{o}{=} \PY{n}{pd}\PY{o}{.}\PY{n}{read\PYZus{}csv}\PY{p}{(}\PY{l+s+s1}{\PYZsq{}}\PY{l+s+s1}{datasets/ml\PYZhy{}latest\PYZhy{}small/ratings.csv}\PY{l+s+s1}{\PYZsq{}}\PY{p}{)}
        \PY{n}{ratings\PYZus{}df}\PY{o}{.}\PY{n}{columns} \PY{o}{=} \PY{p}{[}\PY{l+s+s1}{\PYZsq{}}\PY{l+s+s1}{user\PYZus{}id}\PY{l+s+s1}{\PYZsq{}}\PY{p}{,} \PY{l+s+s1}{\PYZsq{}}\PY{l+s+s1}{movie\PYZus{}id}\PY{l+s+s1}{\PYZsq{}}\PY{p}{,} \PY{l+s+s1}{\PYZsq{}}\PY{l+s+s1}{rating}\PY{l+s+s1}{\PYZsq{}}\PY{p}{,} \PY{l+s+s1}{\PYZsq{}}\PY{l+s+s1}{timestamp}\PY{l+s+s1}{\PYZsq{}}\PY{p}{]}

        \PY{n}{user\PYZus{}ratings} \PY{o}{=} \PY{n}{ratings\PYZus{}df}\PY{o}{.}\PY{n}{pivot}\PY{p}{(}\PY{n}{index}\PY{o}{=}\PY{l+s+s1}{\PYZsq{}}\PY{l+s+s1}{movie\PYZus{}id}\PY{l+s+s1}{\PYZsq{}}\PY{p}{,} \PY{n}{columns}\PY{o}{=}\PY{l+s+s1}{\PYZsq{}}\PY{l+s+s1}{user\PYZus{}id}\PY{l+s+s1}{\PYZsq{}}\PY{p}{,}
                                        \PY{n}{values}\PY{o}{=}\PY{l+s+s1}{\PYZsq{}}\PY{l+s+s1}{rating}\PY{l+s+s1}{\PYZsq{}}\PY{p}{)}
\end{Verbatim}


    \begin{Verbatim}[commandchars=\\\{\}]
{\color{incolor}In [{\color{incolor}2}]:} \PY{n}{ui\PYZus{}matrix} \PY{o}{=} \PY{n}{np}\PY{o}{.}\PY{n}{copy}\PY{p}{(}\PY{n}{user\PYZus{}ratings}\PY{o}{.}\PY{n}{values}\PY{p}{)}
        \PY{n}{popularity} \PY{o}{=} \PY{n}{np}\PY{o}{.}\PY{n}{isfinite}\PY{p}{(}\PY{n}{ui\PYZus{}matrix}\PY{p}{)}\PY{o}{.}\PY{n}{astype}\PY{p}{(}\PY{n+nb}{int}\PY{p}{)}\PY{o}{.}\PY{n}{sum}\PY{p}{(}\PY{n}{axis}\PY{o}{=}\PY{l+m+mi}{1}\PY{p}{)}
        \PY{n}{means} \PY{o}{=} \PY{n}{np}\PY{o}{.}\PY{n}{nanmean}\PY{p}{(}\PY{n}{ui\PYZus{}matrix}\PY{p}{,} \PY{n}{axis}\PY{o}{=}\PY{l+m+mi}{1}\PY{p}{)}
        \PY{n}{ui\PYZus{}matrix} \PY{o}{=} \PY{n}{ui\PYZus{}matrix} \PY{o}{\PYZhy{}} \PY{n}{means}\PY{o}{.}\PY{n}{reshape}\PY{p}{(}\PY{o}{\PYZhy{}}\PY{l+m+mi}{1}\PY{p}{,} \PY{l+m+mi}{1}\PY{p}{)}
        \PY{n}{ui\PYZus{}matrix} \PY{o}{=} \PY{n}{np}\PY{o}{.}\PY{n}{nan\PYZus{}to\PYZus{}num}\PY{p}{(}\PY{n}{ui\PYZus{}matrix}\PY{p}{)}
\end{Verbatim}


Hacemos nuestra descomposición matricial especificando 500 caracteristicas.

    \begin{Verbatim}[commandchars=\\\{\}]
{\color{incolor}In [{\color{incolor}3}]:} \PY{k+kn}{from} \PY{n+nn}{scipy}\PY{n+nn}{.}\PY{n+nn}{sparse}\PY{n+nn}{.}\PY{n+nn}{linalg} \PY{k}{import} \PY{n}{svds}

        \PY{n}{u}\PY{p}{,} \PY{n}{s}\PY{p}{,} \PY{n}{vt} \PY{o}{=} \PY{n}{svds}\PY{p}{(}\PY{n}{ui\PYZus{}matrix}\PY{p}{,} \PY{n}{k}\PY{o}{=}\PY{l+m+mi}{500}\PY{p}{)}
\end{Verbatim}

Seleccionamos un usuario aleatorio.

    \begin{Verbatim}[commandchars=\\\{\}]
{\color{incolor}In [{\color{incolor}4}]:} \PY{n}{user} \PY{o}{=} \PY{n}{user\PYZus{}ratings}\PY{o}{.}\PY{n}{sample}\PY{p}{(}\PY{n}{axis}\PY{o}{=}\PY{l+m+mi}{1}\PY{p}{)}
        \PY{n}{user\PYZus{}id} \PY{o}{=} \PY{n}{user}\PY{o}{.}\PY{n}{columns}\PY{p}{[}\PY{l+m+mi}{0}\PY{p}{]}
        \PY{n}{user} \PY{o}{=} \PY{n}{user}\PY{o}{.}\PY{n}{assign}\PY{p}{(}\PY{n}{title}\PY{o}{=}\PY{n}{movies}\PY{o}{.}\PY{n}{title}\PY{p}{[}\PY{n}{user\PYZus{}ratings}\PY{o}{.}\PY{n}{index}\PY{p}{]}\PY{p}{)}
        \PY{n}{user}\PY{o}{.}\PY{n}{columns} \PY{o}{=} \PY{p}{[}\PY{l+s+s1}{\PYZsq{}}\PY{l+s+s1}{rating}\PY{l+s+s1}{\PYZsq{}}\PY{p}{,} \PY{l+s+s1}{\PYZsq{}}\PY{l+s+s1}{title}\PY{l+s+s1}{\PYZsq{}}\PY{p}{]}
        \PY{n}{user}\PY{o}{.}\PY{n}{sort\PYZus{}values}\PY{p}{(}\PY{n}{by}\PY{o}{=}\PY{l+s+s1}{\PYZsq{}}\PY{l+s+s1}{rating}\PY{l+s+s1}{\PYZsq{}}\PY{p}{,} \PY{n}{ascending}\PY{o}{=}\PY{k+kc}{False}\PY{p}{)}\PY{o}{.}\PY{n}{head}\PY{p}{(}\PY{l+m+mi}{20}\PY{p}{)}
\end{Verbatim}


\begin{Verbatim}[commandchars=\\\{\}]
{\color{outcolor}Out[{\color{outcolor}4}]:}           rating                                              title
        movie\_id
        1            5.0                                   Toy Story (1995)
        592          5.0                                      Batman (1989)
        2858         5.0                             American Beauty (1999)
        2797         5.0                                         Big (1988)
        3114         5.0                                 Toy Story 2 (1999)
        3176         5.0                    Talented Mr. Ripley, The (1999)
        1639         5.0                                 Chasing Amy (1997)
        3481         5.0                               High Fidelity (2000)
        3578         5.0                                   Gladiator (2000)
        1356         5.0                    Star Trek: First Contact (1996)
        1259         5.0                                 Stand by Me (1986)
        1221         5.0                     Godfather: Part II, The (1974)
        1214         5.0                                       Alien (1979)
        1198         5.0  Raiders of the Lost Ark (Indiana Jones and the{\ldots}
        1193         5.0             One Flew Over the Cuckoo's Nest (1975)
        4306         5.0                                       Shrek (2001)
        2918         5.0                    Ferris Bueller's Day Off (1986)
        5952         5.0      Lord of the Rings: The Two Towers, The (2002)
        4846         5.0  Iron Monkey (Siu nin Wong Fei-hung ji: Tit Ma {\ldots}
        260          5.0          Star Wars: Episode IV - A New Hope (1977)
\end{Verbatim}

Calculamos nuestras calificaciones aproximadas.

    \begin{Verbatim}[commandchars=\\\{\}]
{\color{incolor}In [{\color{incolor}5}]:} \PY{n}{theta\PYZus{}df} \PY{o}{=} \PY{n}{pd}\PY{o}{.}\PY{n}{DataFrame}\PY{p}{(}\PY{n}{vt}\PY{o}{.}\PY{n}{T}\PY{p}{,} \PY{n}{index}\PY{o}{=}\PY{n}{user\PYZus{}ratings}\PY{o}{.}\PY{n}{columns}\PY{p}{)}

        \PY{n}{user\PYZus{}theta} \PY{o}{=} \PY{n}{theta\PYZus{}df}\PY{o}{.}\PY{n}{loc}\PY{p}{[}\PY{n}{user\PYZus{}id}\PY{p}{]}

        \PY{n}{pred} \PY{o}{=} \PY{p}{(}\PY{n}{user\PYZus{}theta}\PY{o}{.}\PY{n}{values} \PY{o}{@} \PY{p}{(}\PY{n}{s} \PY{o}{*} \PY{n}{u}\PY{p}{)}\PY{o}{.}\PY{n}{T}\PY{p}{)} \PY{o}{+} \PY{n}{means}
\end{Verbatim}

Imprimimos los resultados.

    \begin{Verbatim}[commandchars=\\\{\}]
{\color{incolor}In [{\color{incolor}6}]:} \PY{n}{user}\PY{p}{[}\PY{l+s+s1}{\PYZsq{}}\PY{l+s+s1}{predicted}\PY{l+s+s1}{\PYZsq{}}\PY{p}{]} \PY{o}{=} \PY{n}{pred}
        \PY{n}{user}\PY{p}{[}\PY{l+s+s1}{\PYZsq{}}\PY{l+s+s1}{relevance}\PY{l+s+s1}{\PYZsq{}}\PY{p}{]} \PY{o}{=} \PY{n}{user}\PY{p}{[}\PY{l+s+s1}{\PYZsq{}}\PY{l+s+s1}{predicted}\PY{l+s+s1}{\PYZsq{}}\PY{p}{]} \PY{o}{*} \PY{n}{popularity}
        \PY{n}{user}\PY{p}{[}\PY{n}{user}\PY{o}{.}\PY{n}{rating}\PY{o}{.}\PY{n}{isnull}\PY{p}{(}\PY{p}{)}\PY{p}{]}\PY{o}{.}\PY{n}{sort\PYZus{}values}\PY{p}{(}\PY{n}{by}\PY{o}{=}\PY{l+s+s1}{\PYZsq{}}\PY{l+s+s1}{relevance}\PY{l+s+s1}{\PYZsq{}}\PY{p}{,} \PY{n}{ascending}\PY{o}{=}\PY{k+kc}{False}\PY{p}{)}\PY{o}{.}\PY{n}{head}\PY{p}{(}\PY{l+m+mi}{20}\PY{p}{)}
\end{Verbatim}


\begin{Verbatim}[commandchars=\\\{\}]
{\color{outcolor}Out[{\color{outcolor}6}]:}           rating                                              title  \textbackslash{}
        movie\_id
        296          NaN                                Pulp Fiction (1994)
        356          NaN                                Forrest Gump (1994)
        527          NaN                            Schindler's List (1993)
        1196         NaN  Star Wars: Episode V - The Empire Strikes Back{\ldots}
        608          NaN                                       Fargo (1996)
        589          NaN                  Terminator 2: Judgment Day (1991)
        1270         NaN                          Back to the Future (1985)
        110          NaN                                  Braveheart (1995)
        858          NaN                              Godfather, The (1972)
        1210         NaN  Star Wars: Episode VI - Return of the Jedi (1983)
        50           NaN                         Usual Suspects, The (1995)
        47           NaN                        Seven (a.k.a. Se7en) (1995)
        588          NaN                                     Aladdin (1992)
        32           NaN          Twelve Monkeys (a.k.a. 12 Monkeys) (1995)
        780          NaN               Independence Day (a.k.a. ID4) (1996)
        364          NaN                              Lion King, The (1994)
        2028         NaN                         Saving Private Ryan (1998)
        590          NaN                          Dances with Wolves (1990)
        1580         NaN                   Men in Black (a.k.a. MIB) (1997)
        1197         NaN                         Princess Bride, The (1987)

                  predicted    relevance
        movie\_id
        296        4.248962  1376.663713
        356        4.032926  1375.227797
        527        4.304056  1050.189678
        1196       4.237824   991.650929
        608        4.257189   953.610249
        589        4.005997   949.421358
        1270       4.014198   907.208763
        110        3.937018   897.640189
        858        4.465236   893.047173
        1210       4.057057   880.381442
        50         4.344517   873.247952
        47         4.026175   809.261176
        588        3.662208   787.374662
        32         3.928926   770.069566
        780        3.487695   760.317444
        364        3.794904   758.980718
        2028       3.934426   751.475424
        590        3.707650   748.945372
        1580       3.671700   697.622927
        1197       4.206574   685.671613
\end{Verbatim}

\subsection{Recomendaciones iniciales con clustering en MovieLens}
\label{subsec:a6}

Se presenta una implementacion del algoritmo de recomendaciones iniciales
usando clustering descrito en la seccion \ref{clustering} para una matriz de
contenido obtenida por factorización SVD usando la base de datos de \emph{MovieLens}.

\begin{Verbatim}[commandchars=\\\{\}]
{\color{incolor}In [{\color{incolor}1}]:} \PY{k+kn}{import} \PY{n+nn}{pandas} \PY{k}{as} \PY{n+nn}{pd}
    \PY{k+kn}{import} \PY{n+nn}{numpy} \PY{k}{as} \PY{n+nn}{np}

    \PY{n}{movies} \PY{o}{=} \PY{n}{pd}\PY{o}{.}\PY{n}{read\PYZus{}csv}\PY{p}{(}\PY{l+s+s1}{\PYZsq{}}\PY{l+s+s1}{datasets/ml\PYZhy{}latest\PYZhy{}small/movies.csv}\PY{l+s+s1}{\PYZsq{}}\PY{p}{,} \PY{n}{index\PYZus{}col}\PY{o}{=}\PY{l+m+mi}{0}\PY{p}{)}
    \PY{n}{ratings\PYZus{}df} \PY{o}{=} \PY{n}{pd}\PY{o}{.}\PY{n}{read\PYZus{}csv}\PY{p}{(}\PY{l+s+s1}{\PYZsq{}}\PY{l+s+s1}{datasets/ml\PYZhy{}latest\PYZhy{}small/ratings.csv}\PY{l+s+s1}{\PYZsq{}}\PY{p}{)}
    \PY{n}{ratings\PYZus{}df}\PY{o}{.}\PY{n}{columns} \PY{o}{=} \PY{p}{[}\PY{l+s+s1}{\PYZsq{}}\PY{l+s+s1}{user\PYZus{}id}\PY{l+s+s1}{\PYZsq{}}\PY{p}{,} \PY{l+s+s1}{\PYZsq{}}\PY{l+s+s1}{movie\PYZus{}id}\PY{l+s+s1}{\PYZsq{}}\PY{p}{,} \PY{l+s+s1}{\PYZsq{}}\PY{l+s+s1}{rating}\PY{l+s+s1}{\PYZsq{}}\PY{p}{,} \PY{l+s+s1}{\PYZsq{}}\PY{l+s+s1}{timestamp}\PY{l+s+s1}{\PYZsq{}}\PY{p}{]}

    \PY{n}{user\PYZus{}ratings} \PY{o}{=} \PY{n}{ratings\PYZus{}df}\PY{o}{.}\PY{n}{pivot}\PY{p}{(}\PY{n}{index}\PY{o}{=}\PY{l+s+s1}{\PYZsq{}}\PY{l+s+s1}{movie\PYZus{}id}\PY{l+s+s1}{\PYZsq{}}\PY{p}{,} \PY{n}{columns}\PY{o}{=}\PY{l+s+s1}{\PYZsq{}}\PY{l+s+s1}{user\PYZus{}id}\PY{l+s+s1}{\PYZsq{}}\PY{p}{,}
                                    \PY{n}{values}\PY{o}{=}\PY{l+s+s1}{\PYZsq{}}\PY{l+s+s1}{rating}\PY{l+s+s1}{\PYZsq{}}\PY{p}{)}
\end{Verbatim}


\begin{Verbatim}[commandchars=\\\{\}]
{\color{incolor}In [{\color{incolor}2}]:} \PY{n}{ui\PYZus{}matrix} \PY{o}{=} \PY{n}{np}\PY{o}{.}\PY{n}{copy}\PY{p}{(}\PY{n}{user\PYZus{}ratings}\PY{o}{.}\PY{n}{values}\PY{p}{)}
    \PY{n}{popularity} \PY{o}{=} \PY{n}{np}\PY{o}{.}\PY{n}{isfinite}\PY{p}{(}\PY{n}{ui\PYZus{}matrix}\PY{p}{)}\PY{o}{.}\PY{n}{astype}\PY{p}{(}\PY{n+nb}{int}\PY{p}{)}\PY{o}{.}\PY{n}{sum}\PY{p}{(}\PY{n}{axis}\PY{o}{=}\PY{l+m+mi}{1}\PY{p}{)}
    \PY{n}{means} \PY{o}{=} \PY{n}{np}\PY{o}{.}\PY{n}{nanmean}\PY{p}{(}\PY{n}{ui\PYZus{}matrix}\PY{p}{,} \PY{n}{axis}\PY{o}{=}\PY{l+m+mi}{1}\PY{p}{)}
    \PY{n}{ui\PYZus{}matrix} \PY{o}{=} \PY{n}{ui\PYZus{}matrix} \PY{o}{\PYZhy{}} \PY{n}{means}\PY{o}{.}\PY{n}{reshape}\PY{p}{(}\PY{o}{\PYZhy{}}\PY{l+m+mi}{1}\PY{p}{,} \PY{l+m+mi}{1}\PY{p}{)}
    \PY{n}{ui\PYZus{}matrix} \PY{o}{=} \PY{n}{np}\PY{o}{.}\PY{n}{nan\PYZus{}to\PYZus{}num}\PY{p}{(}\PY{n}{ui\PYZus{}matrix}\PY{p}{)}
\end{Verbatim}


\begin{Verbatim}[commandchars=\\\{\}]
{\color{incolor}In [{\color{incolor}3}]:} \PY{k+kn}{from} \PY{n+nn}{scipy}\PY{n+nn}{.}\PY{n+nn}{sparse}\PY{n+nn}{.}\PY{n+nn}{linalg} \PY{k}{import} \PY{n}{svds}

    \PY{n}{u}\PY{p}{,} \PY{n}{s}\PY{p}{,} \PY{n}{vt} \PY{o}{=} \PY{n}{svds}\PY{p}{(}\PY{n}{ui\PYZus{}matrix}\PY{p}{,} \PY{n}{k}\PY{o}{=}\PY{l+m+mi}{500}\PY{p}{)}
\end{Verbatim}

Ahora realizamos clustering K medias para encontrar clusters de películas
similares. El numero de clusters es una situación delicada. Necesitamos un
numero de clusters que divida los artículos representativamente, pero tampoco
puede ser muy bajo, ya que la idea es tener al usuario calificar un articulo de
cada cluster.

\begin{Verbatim}[commandchars=\\\{\}]
{\color{incolor}In [{\color{incolor}4}]:} \PY{k+kn}{from} \PY{n+nn}{sklearn}\PY{n+nn}{.}\PY{n+nn}{cluster} \PY{k}{import} \PY{n}{KMeans}

    \PY{n}{n\PYZus{}clusters} \PY{o}{=} \PY{l+m+mi}{10}
    \PY{n}{kmeans} \PY{o}{=} \PY{n}{KMeans}\PY{p}{(}\PY{n}{n\PYZus{}clusters}\PY{p}{)}
    \PY{n}{clusters} \PY{o}{=} \PY{n}{kmeans}\PY{o}{.}\PY{n}{fit\PYZus{}predict}\PY{p}{(}\PY{n}{u}\PY{p}{)}

    \PY{n}{cluster\PYZus{}mask} \PY{o}{=} \PY{n}{np}\PY{o}{.}\PY{n}{asarray}\PY{p}{(}\PY{p}{[}\PY{n}{clusters} \PY{o}{==} \PY{n}{i} \PY{k}{for} \PY{n}{i} \PY{o+ow}{in} \PY{n+nb}{range}\PY{p}{(}\PY{n}{n\PYZus{}clusters}\PY{p}{)}\PY{p}{]}\PY{p}{)}

    \PY{n}{ratings\PYZus{}mask} \PY{o}{=} \PY{n}{np}\PY{o}{.}\PY{n}{isfinite}\PY{p}{(}\PY{n}{user\PYZus{}ratings}\PY{o}{.}\PY{n}{values}\PY{p}{)}
\end{Verbatim}


Lo siguiente es obtener el articulo mas relevante de cada cluster. Para hacer
esto, primero necesitamos una métrica de relevancia para nuestros articulos.
Inspirandonos en el TFIDF, definimos la relevancia de un articulo $i$ en funcion
sus calificaciones existentes $R_i$ como

$$t_i = \sum\limits_{j\in U}R_{i, j}log\big(1 + \frac{N}{n_i}\big)$$

Usando esta métrica, calculamos la relevancia de todos los artículos y ordenamos los objetos de nuestros clusters usándola.

\begin{Verbatim}[commandchars=\\\{\}]
{\color{incolor}In [{\color{incolor}5}]:} \PY{n}{np}\PY{o}{.}\PY{n}{sum}\PY{p}{(}\PY{n}{ratings\PYZus{}mask}\PY{p}{,} \PY{n}{axis}\PY{o}{=}\PY{l+m+mi}{1}\PY{p}{)}\PY{o}{.}\PY{n}{size}
    \PY{n}{relevance} \PY{o}{=} \PY{p}{(}\PY{p}{(}\PY{n}{np}\PY{o}{.}\PY{n}{sum}\PY{p}{(}\PY{n}{ui\PYZus{}matrix}\PY{p}{,} \PY{n}{axis}\PY{o}{=}\PY{l+m+mi}{1}\PY{p}{)} \PY{o}{/} \PY{n}{user\PYZus{}ratings}\PY{o}{.}\PY{n}{shape}\PY{p}{[}\PY{l+m+mi}{1}\PY{p}{]}\PY{p}{)} \PY{o}{*}
                 \PY{p}{(}\PY{n}{user\PYZus{}ratings}\PY{o}{.}\PY{n}{shape}\PY{p}{[}\PY{l+m+mi}{0}\PY{p}{]} \PY{o}{/} \PY{n}{np}\PY{o}{.}\PY{n}{sum}\PY{p}{(}\PY{n}{ratings\PYZus{}mask}\PY{p}{,} \PY{n}{axis}\PY{o}{=}\PY{l+m+mi}{1}\PY{p}{)}\PY{p}{)}\PY{p}{)}

    \PY{n}{relevance\PYZus{}df} \PY{o}{=} \PY{n}{pd}\PY{o}{.}\PY{n}{DataFrame}\PY{p}{(}\PY{n}{relevance}\PY{p}{,} \PY{n}{index}\PY{o}{=}\PY{n}{user\PYZus{}ratings}\PY{o}{.}\PY{n}{index}\PY{p}{)}
    \PY{n}{relevance\PYZus{}df}\PY{p}{[}\PY{l+s+s1}{\PYZsq{}}\PY{l+s+s1}{title}\PY{l+s+s1}{\PYZsq{}}\PY{p}{]} \PY{o}{=} \PY{p}{[}\PY{n}{movies}\PY{o}{.}\PY{n}{title}\PY{p}{[}\PY{n+nb}{id}\PY{p}{]} \PY{k}{for} \PY{n+nb}{id} \PY{o+ow}{in} \PY{n}{relevance\PYZus{}df}\PY{o}{.}\PY{n}{index}\PY{p}{]}
    \PY{n}{relevance\PYZus{}df}\PY{o}{.}\PY{n}{columns} \PY{o}{=} \PY{p}{[}\PY{l+s+s1}{\PYZsq{}}\PY{l+s+s1}{relevance}\PY{l+s+s1}{\PYZsq{}}\PY{p}{,} \PY{l+s+s1}{\PYZsq{}}\PY{l+s+s1}{title}\PY{l+s+s1}{\PYZsq{}}\PY{p}{]}

    \PY{n}{relevance\PYZus{}df}\PY{o}{.}\PY{n}{sort\PYZus{}values}\PY{p}{(}\PY{n}{by}\PY{o}{=}\PY{l+s+s1}{\PYZsq{}}\PY{l+s+s1}{relevance}\PY{l+s+s1}{\PYZsq{}}\PY{p}{,} \PY{n}{ascending}\PY{o}{=}\PY{k+kc}{False}\PY{p}{)}


    \PY{n}{masked\PYZus{}array} \PY{o}{=} \PY{n}{np}\PY{o}{.}\PY{n}{tile}\PY{p}{(}\PY{n}{relevance}\PY{p}{,} \PY{p}{(}\PY{n}{cluster\PYZus{}mask}\PY{o}{.}\PY{n}{shape}\PY{p}{[}\PY{l+m+mi}{0}\PY{p}{]}\PY{p}{,} \PY{l+m+mi}{1}\PY{p}{)}\PY{p}{)}
    \PY{n}{masked\PYZus{}array}\PY{p}{[}\PY{o}{\PYZti{}}\PY{n}{cluster\PYZus{}mask}\PY{p}{]} \PY{o}{=} \PY{o}{\PYZhy{}}\PY{n}{np}\PY{o}{.}\PY{n}{inf}
    \PY{n}{sorted\PYZus{}array} \PY{o}{=} \PY{n}{np}\PY{o}{.}\PY{n}{argsort}\PY{p}{(}\PY{n}{masked\PYZus{}array}\PY{p}{,} \PY{n}{axis}\PY{o}{=}\PY{l+m+mi}{1}\PY{p}{)}

    \PY{n}{relevance\PYZus{}df}\PY{o}{.}\PY{n}{iloc}\PY{p}{[}\PY{n}{sorted\PYZus{}array}\PY{p}{[}\PY{p}{:}\PY{p}{,} \PY{o}{\PYZhy{}}\PY{l+m+mi}{1}\PY{p}{]}\PY{p}{]}
\end{Verbatim}

Obtenemos los articulos mas relevantes de cada cluster. Estas seran las que
recomendaremos al usuario.

\begin{Verbatim}[commandchars=\\\{\}]
{\color{outcolor}Out[{\color{outcolor}5}]:}              relevance                                              title
    movie\_id
    1304      5.920166e-15          Butch Cassidy and the Sundance Kid (1969)
    4447     -2.042610e-15                              Legally Blonde (2001)
    3354      8.000224e-16                             Mission to Mars (2000)
    344      -4.114401e-16                  Ace Ventura: Pet Detective (1994)
    778      -5.806614e-15                               Trainspotting (1996)
    3730      5.739291e-15                           Conversation, The (1974)
    780      -5.504741e-16               Independence Day (a.k.a. ID4) (1996)
    25       -2.613935e-15                           Leaving Las Vegas (1995)
    7153     -1.159123e-15  Lord of the Rings: The Return of the King, The{\ldots}
    1196     -8.205358e-16  Star Wars: Episode V - The Empire Strikes Back{\ldots}
\end{Verbatim}

\end{document}
